\documentclass[a4paper,11pt,twoside,openright]{memoir}

% rubber: module index

%BEGIN LATEX
\usepackage{comment}
\newcommand{\ahref}[2]{{#2}}
\excludecomment{htmlonly}
\newenvironment{latexonly}{}{}
%END LATEX

%HEVEA\@addimagenopt{-pdf}

%BEGIN LATEX
% tells memoir style to number subsections
\setsecnumdepth{subsection}
%END LATEX

\usepackage[T1]{fontenc}
\usepackage{lmodern}
%\usepackage{url}
\usepackage[pdftex,colorlinks=true,urlcolor=blue,pdfstartview=FitH]{hyperref}

%BEGIN LATEX
\usepackage{upquote}
%END LATEX

%BEGIN LATEX
\usepackage{graphicx}
%END LATEX
%HEVEA \newcommand{\includegraphics}[2][2]{\imgsrc{#2}}

\usepackage{listings}
\usepackage{xspace}

%BEGIN LATEX
\setulmarginsandblock{30mm}{30mm}{*}
\setlrmarginsandblock{30mm}{30mm}{*}
\setheadfoot{15pt}{38pt}
\checkandfixthelayout

% placement of figures
\renewcommand{\textfraction}{0.01}
\renewcommand{\topfraction}{0.99}
\renewcommand{\bottomfraction}{0.99}
%END LATEX
\setcounter{topnumber}{4}
\setcounter{bottomnumber}{4}
\setcounter{totalnumber}{8}

%HEVEA \newstyle{table.lstframe}{width:100\%;border-width:1px;}

% \usepackage[toc,nonumberlist]{glossaries}
% \makeglossaries

% \usepackage{glossary}
% \makeglossary
% \glossary{name={entry name}, description={entry description}}

% for ocamldoc generated pages
%\usepackage{ocamldoc}
%\let\tt\ttfamily
%\let\bf\bfseries

\usepackage{ifthen}
\input{./macros.tex}
\input{./replayer_macros.tex}

\input{./version.tex}

\makeindex

%HEVEA\title{The Why3 platform}

\begin{document}
\sloppy
%BEGIN LATEX
\hbadness=5000
%END LATEX

\thispagestyle{empty}

\begin{center}

%BEGIN LATEX
\rule\textwidth{0.8mm}
%END LATEX

\vfill

{
%BEGIN LATEX
\fontsize{40}{40pt}\selectfont
%END LATEX
%HEVEA \Huge
\bfseries\sffamily The Why3 platform}

\vfill

%BEGIN LATEX
\rule\textwidth{0.8mm}
%END LATEX

\vfill

% \todo{NE PAS DISTRIBUER TANT QU'IL RESTE DES TODOS}

%BEGIN LATEX
\begin{LARGE}
%END LATEX
  Version \whyversion{}, December 2013
%BEGIN LATEX
\end{LARGE}
%END LATEX

\vfill

%BEGIN LATEX
\begin{Large}
%END LATEX
  \begin{tabular}{c}
  Fran\c{c}ois Bobot$^{1,2}$ \\
  Jean-Christophe Filli\^atre$^{1,2}$  \\
  Claude March\'e$^{2,1}$ \\
  Guillaume Melquiond$^{2,1}$\\
  Andrei Paskevich$^{1,2}$
\end{tabular}
%BEGIN LATEX
\end{Large}
%END LATEX
\vfill

\begin{flushleft}

\begin{tabular}{l}
$^1$ LRI, CNRS \& University Paris-Sud, Orsay, F-91405 \\
$^2$ Inria Saclay -- \^Ile-de-France, Palaiseau, F-91120
\end{tabular}

%BEGIN LATEX
\bigskip
%END LATEX

  \textcopyright 2010-2013 University Paris-Sud, CNRS, Inria

\urldef{\urlutcat}{\url}{http://frama-c.com/u3cat/}
\urldef{\urlhilite}{\url}{http://www.open-do.org/projects/hi-lite/}

  This work has been partly supported by the `\ahref{\urlutcat}{U3CAT}'
  national ANR project (ANR-08-SEGI-021-08\begin{latexonly},
  \urlutcat\end{latexonly}) and by the `\ahref{\urlhilite}{Hi-Lite}'
  \begin{latexonly}(\urlhilite)\end{latexonly} FUI project of the
  System@tic competitivity cluster.

\end{flushleft}
\end{center}

\chapter*{Foreword}

%This is the manual for the Why platform version 3, or \why for
%short.
\why is a platform for deductive program verification. It provides
a rich language for specification and programming, called \whyml, and
relies on external theorem provers, both automated and interactive,
to discharge verification conditions. \why comes with a standard
library of logical theories (integer and real arithmetic, Boolean
operations, sets and maps, etc.) and basic programming data structures
(arrays, queues, hash tables, etc.). A user can write \whyml programs
directly and get correct-by-construction OCaml programs through an
automated extraction mechanism. \whyml is also used as an intermediate
language for the verification of C, Java, or Ada programs.

\why is a complete reimplementation %~\cite{boogie11why3}
of the former Why platform~\cite{filliatre07cav}.
%for program verification.
Among the new features are: numerous
extensions to the input language, a new architecture for calling
external provers, and a well-designed API, allowing to use \why as a
software library.  An important emphasis is put on modularity and
genericity, giving the end user a possibility to easily reuse \why
formalizations or to add support for a new external prover if wanted.

\subsection*{Availability}

\why project page is \url{http://why3.lri.fr/}.  The last distribution
is available there, in source format, together with this documentation
and several examples.

\why is distributed as open source and freely available under the
terms of the GNU LGPL 2.1. See the file \texttt{LICENSE}.

See the file \texttt{INSTALL} for quick installation instructions, and
Section~\ref{sec:install} of this document for more detailed
instructions.

\subsection*{Contact}

There is a public mailing list for users' discussions:
\url{http://lists.gforge.inria.fr/mailman/listinfo/why3-club}.

Report any bug to the \why Bug Tracking System:
\url{https://gforge.inria.fr/tracker/?atid=10293&group_id=2990&func=browse}.


\subsection*{Acknowledgements}

We gratefully thank the people who contributed to \why, directly or
indirectly: Romain Bardou, Stefan Berghofer, Sylvie Boldo, Martin
Clochard, Simon Cruanes, Leon Gondelman, Johannes Kanig, St\'ephane
Lescuyer, David Mentr\'e, Sim\~ao Melo de Sousa, Benjamin Monate,
Thi-Minh-Tuyen Nguyen, Asma Tafat, Piotr Trojanek.


\cleardoublepage

%BEGIN LATEX
\tableofcontents
%END LATEX

%
\chapter{Introduction}

\section{Architecture and Terminology}

Everything in \why revolves around the notion of
\emph{task}\index{task}.  \why, as a platform, is a tool that
translates its input to a number of tasks, and dispatches these tasks
to external provers. 

Essentially, a task is a sequence of premises followed by a goal
(\ie a \emph{logical sequent} with exactly one formula in the
succedent). The language of tasks is based on first-order language
extended with 
\begin{itemize}
\item polymorphic types;
\item algebraic types together with pattern matching;
\item definitions of functions and predicates, possibly recursive or
  mutually recursive;
\item inductively defined predicates;
\item \texttt{let} and \texttt{if-then-else} constructs;
%\item Hilbert's epsilon construct
\end{itemize}

% \todo{continue}

\section{Organization of this document}

This document is organized as follows. The first part, made of three chapters, provides tutorials to learn how to use \why. The second part gathers reference
manuals, giving detailed technical informations.

Chapter~\ref{chap:starting} explains how to use the GUI
for visualizing theories and goals, calling external provers for
trying to solve them, and applying transformations to simplify
them. It also presents the basic use of \why in
batch. Chapter~\ref{chap:syntax} presents the input syntax for file
defining \why theories. The semantics is given informally with
examples.  The two first chapters are thus to read for the beginners.

%Chapter~\ref{chap:whyml} presents the
%verification condition generator built upon the Why3 core.  
% The two
% next chapters are for users with a little more experience, in
% particular those who wants to use Why for verification of algorithms.

Chapter~\ref{chap:api} presents how to use \why programmatically,
using the API.  It is for the experimented users, who wants to link
\why library with their own code.

Part 2 provides: 
\begin{itemize}
\item In Chapter~\ref{chap:syntaxref}, the input syntax of files.
% \item In Chapter~\ref{chap:library}, the standard library of
%   theories distributed with \why.
\item In Chapter~\ref{chap:manpages}, the technical manual pages for the
  tools of the platform. All tool options, and all the configuration
  files are described in details there.
% \item In Chapter~\ref{chap:apidoc}, the technical documentation of the API.
\end{itemize}


%%% Local Variables:
%%% mode: latex
%%% TeX-PDF-mode: t
%%% TeX-master: "manual"
%%% End:


\part{Tutorial}

\chapter{Getting Started}
\label{chap:starting}

\section{Hello Proofs}

The first step in using \why is to write a suitable input
file. When one wants to learn a programming language, one starts by
writing a basic program. Here is our first \why file, which is the file
\texttt{examples/logic/hello\_proof.why} of the distribution.
It contains a small set of goals.
\lstinputlisting[language=why3]{../examples/logic/hello_proof.why}

Any declaration must occur
inside a theory, which is in that example called HelloProof and
labeled with a comment inside double quotes. It contains three goals
named $G_1,G_2,G_3$. The first two are basic propositional goals,
whereas the third involves some integer arithmetic, and thus it
requires to import the theory of integer arithmetic from the \why
standard library, which is done by the \texttt{use} declaration above.

We don't give more details here about the syntax and refer to
Chapter~\ref{chap:syntax} for detailed explanations. In the following,
we show how this file is handled in the \why GUI
(Section~\ref{sec:gui}) then in batch mode using the \texttt{why3}
executable (Section~\ref{sec:batch}).


\section{Getting Started with the GUI}
\label{sec:gui}

The graphical interface allows to browse into a file or a set of
files, and check the validity of goals with external provers, in a
friendly way. This section presents the basic use of this GUI. Please
refer to Section~\ref{sec:ideref} for a more complete description.

\begin{figure}[tbp]
%HEVEA\centering
  \includegraphics[width=\textwidth]{gui-0-70-1.png}
  \caption{The GUI when started the very first time}
  \label{fig:gui1}
\end{figure}

The GUI is launched on the file above as follows.
\begin{verbatim}
why3 ide hello_proof.why
\end{verbatim}
When the GUI is started for the first time, you should get a window
that looks like the screenshot of Figure~\ref{fig:gui1}.

The left column is a tool bar which provides different actions to
apply on goals. The section ``Provers'' displays the provers that were
detected as installed on your computer.%
%BEGIN LATEX
\footnote{If not done yet, you
  must perform prover autodetection using \texttt{why3 config
    -{}-detect-provers}}
%END LATEX
%HEVEA {} (If not done yet, you must perform prover autodetection using \texttt{why3 config -{}-detect-provers}.)
Three provers were detected, in this case,
these are Alt-Ergo~\cite{ergo}, Coq~\cite{CoqArt} and
Simplify~\cite{simplify05}.

The middle part is a tree view that
allows to browse inside the theories.
% Initially, the item of this tree
% are closed. We can expand this view using the menu \textsf{View/Expand
%   all} or its shortcut \textsf{Ctrl-E}. This will result is something
% like the screenshot of Figure~\ref{fig:gui2}.
In this tree view, we have a structured view of the file: this file
contains one theory, itself containing three goals.


\begin{figure}[tbp]
%HEVEA\centering
 \includegraphics[width=\textwidth]{gui-0-70-2.png}
  \caption{The GUI with goal G1 selected}
  \label{fig:gui2}
\end{figure}
In Figure~\ref{fig:gui2}, we clicked on the row corresponding to
goal $G_1$. The \emph{task} associated with this goal is then
displayed on the top right, and the corresponding part of the input
file is shown on the bottom right part.


\subsection{Calling provers on goals}

You are now ready to call these provers on the goals. Whenever you
click on a prover button, this prover is called on the goal selected
in the tree view. You can select several goals at a time, either
by using multi-selection (typically by clicking while pressing the
\textsf{Shift} or \textsf{Ctrl} key) or by selecting the parent theory
or the parent file. Let us now select the theory ``HelloProof'' and
click on the \textsf{Simplify} button. After a short time, you should
get the display of Figure~\ref{fig:gui3}.

\begin{figure}[tbp]
%HEVEA\centering
\includegraphics[width=\textwidth]{gui-0-70-3.png}
  \caption{The GUI after Simplify prover is run on each goal}
  \label{fig:gui3}
\end{figure}

Goal $G_1$ is now marked with a green ``checked'' icon in the
status column. This means that the goal is proved by the Simplify
prover. On the contrary, the two other goals are not proved, they remain
marked with an orange question mark.

You can immediately attempt to prove the remaining goals using another
prover, \eg Alt-Ergo, by clicking on the corresponding button.
Goal $G_3$ should be proved now, but not $G_2$.

\subsection{Applying transformations}

Instead of calling a prover on a goal, you can apply a transformation
to it.  Since $G_2$ is a conjunction, a possibility is to split it
into subgoals. You can do that by clicking on the \textsf{Split}
button of section ``Transformations'' of the left toolbar. Now you
have two subgoals, and you can try again a prover on them, for example
Simplify. We already have a lot of goals and proof attempts, so it is a good idea to close the sub-trees which are already proved: this can be done by the menu \textsf{View/Collapse proved goals}, or even better by its shortcut ``Ctrl-C''.
You should see now what is displayed on Figure~\ref{fig:gui4}.

\begin{figure}[tbp]
%HEVEA\centering
 \includegraphics[width=\textwidth]{gui-0-70-4.png}
  \caption{The GUI after splitting goal $G_2$ and collapsing proved goals}
  \label{fig:gui4}
\end{figure}

The first part of goal $G_2$ is still unproved. As a last resort, we
can try to call the Coq proof assistant. The first step is to click on
the \textsf{Coq} button. A new sub-row appear for Coq, and
unsurprisingly the goal is not proved by Coq either. What can be done
now is editing the proof: select that row and then click on the
\textsf{Edit} button in section ``Tools'' of the toolbar. This should
launch the Coq proof editor, which is \texttt{coqide} by default (see
Section~\ref{sec:ideref} for details on how to configure this). You get
now a regular Coq file to fill in, as shown on Figure~\ref{fig:coqide}.
Please be mindful of the comments of this file. They indicate where \why
expects you to fill the blanks. Note that the comments themselves should
not be removed, as they are needed to properly regenerate the file when the
goal is changed. See Section~\ref{sec:coq} for more details.

\begin{figure}[tbp]
%HEVEA\centering
  \includegraphics[width=\textwidth]{coqide-0-81.png}
  \caption{CoqIDE on subgoal 1 of $G_2$}
  \label{fig:coqide}
\end{figure}

Of course, in that particular case, the goal cannot be proved since it
is not valid. The only thing to do is to fix the input file, as
explained below.

\subsection{Modifying the input}

Currently, the GUI does not allow to modify the input file. You must
edit the file external by some editor of your choice. Let us assume we
change the goal $G_2$ by replacing the first occurrence of true by
false, \eg
\begin{whycode}
  goal G2 : (false -> false) /\ (true \/ false)
\end{whycode}
We can reload the modified file in the IDE using menu \textsf{File/Reload}, or the shortcut ``Ctrl-R''. We get the tree view shown on Figure~\ref{fig:gui5}.

\begin{figure}[tbp]
%HEVEA\centering
  \includegraphics[width=\textwidth]{gui-0-70-5.png}
  \caption{File reloaded after modifying goal $G_2$}
  \label{fig:gui5}
\end{figure}

The important feature to notice first is that all the previous proof
attempts and transformations were saved in a database --- an XML file
created when the \why file was opened in the GUI for the first
time. Then, for all the goals that remain unchanged, the previous
proofs are shown again. For the parts that changed, the previous
proofs attempts are shown but marked with ``(obsolete)''\index{obsolete!proof attempt}
so that you
know the results are not accurate. You can now retry to prove all what
remains unproved using any of the provers.

\subsection{Replaying obsolete proofs}

Instead of pushing a prover's button to rerun its proofs, you can
\emph{replay} the existing but obsolete
proof attempts, by clicking on
the \textsf{Replay} button. By default, \textsf{Replay} only replays
proofs that were successful before. If you want to replay all of them,
you must select the context \textsf{all goals} at the top of the left
tool bar.

Notice that replaying can be done in batch mode, using the
\texttt{replay} command (see Section~\ref{sec:why3replayer}) For
example, running the replayer on the \texttt{hello\_proof} example is
as follows (assuming $G_2$ still is
\lstinline|(true -> false) /\ (true \/ false)|).
\begin{verbatim}
$ why3 replay hello_proof
Info: found directory 'hello_proof' for the project
Opening session...[Xml warning] prolog ignored
[Reload] file '../hello_proof.why'
[Reload] theory 'HelloProof'
[Reload] transformation split_goal for goal G2
 done
Progress: 9/9
 2/3
   +--file ../hello_proof.why: 2/3
      +--theory HelloProof: 2/3
         +--goal G2 not proved
Everything OK.
\end{verbatim}
The last line tells us that no differences were detected between the
current run and the run stored in the XML file. The tree above
reminds us that $G_2$ is not proved.

\subsection{Cleaning}

You may want to clean some the proof attempts, \eg removing the
unsuccessful ones when a project is finally fully proved.

A proof or a transformation can be removed by selecting it and
clicking on button \textsf{Remove}. You must confirm the
removal. Beware that there is no way to undo such a removal.

The \textsf{Clean} button performs an automatic removal of all proofs
attempts that are unsuccessful, while there exists a successful proof
attempt for the same goal.

\section{Getting Started with the \why Command}
\label{sec:batch}

The \texttt{prove} command makes it possible to check the validity of goals with external
provers, in batch mode. This section presents the basic use of this
tool. Refer to Section~\ref{sec:why3ref} for a more complete
description of this tool and all its command-line options.

The very first time you want to use \why, you should proceed with
autodetection of external provers. We have already seen how to do
it in the \why GUI. On the command line, this is done as follows
(here ``\texttt{>}'' is the prompt):
\begin{verbatim}
> why3 config --detect
\end{verbatim}
This prints some information messages on what detections are attempted. To know which
provers have been successfully detected, you can do as follows.
\begin{verbatim}
> why3 --list-provers
Known provers:
  alt-ergo (Alt-Ergo)
  coq (Coq)
  simplify (Simplify)
\end{verbatim}
\index{list-provers@\verb+--list-provers+}
The first word of each line is a unique identifier for the associated prover. We thus
have now the three provers Alt-Ergo~\cite{ergo}, Coq~\cite{CoqArt} and
Simplify~\cite{simplify05}.

Let us assume that we want to run Simplify on the HelloProof
example. The command to type and its output are as follows, where the
\verb|-P| option is followed by the unique prover identifier (as shown
by \verb|--list-provers| option).
\begin{verbatim}
> why3 prove -P simplify hello_proof.why
hello_proof.why HelloProof G1 : Valid (0.10s)
hello_proof.why HelloProof G2 : Unknown: Unknown (0.01s)
hello_proof.why HelloProof G3 : Unknown: Unknown (0.00s)
\end{verbatim}
Unlike the \why GUI, the command-line tool does not save the proof attempts
or applied transformations in a database.

We can also specify which goal or goals to prove. This is done by giving
first a theory identifier, then goal identifier(s). Here is the way to
call Alt-Ergo on goals $G_2$ and $G_3$.
\begin{verbatim}
> why3 prove -P alt-ergo hello_proof.why -T HelloProof -G G2 -G G3
hello_proof.why HelloProof G2 : Unknown: Unknown (0.01s)
hello_proof.why HelloProof G3 : Valid (0.01s)
\end{verbatim}

Finally, a transformation to apply to goals before proving them can be
specified. To know the unique identifier associated to
a transformation, do as follows.
\begin{verbatim}
> why3 --list-transforms
Known non-splitting transformations:
  [...]

Known splitting transformations:
  [...]
  split_goal
  split_intro
\end{verbatim}
Here is how you can split the goal $G_2$ before calling
Simplify on the resulting subgoals.
\begin{verbatim}
> why3 prove -P simplify hello_proof.why -a split_goal -T HelloProof -G G2
hello_proof.why HelloProof G2 : Unknown: Unknown (0.00s)
hello_proof.why HelloProof G2 : Valid (0.00s)
\end{verbatim}
Section~\ref{sec:transformations} gives the description of the various
transformations available.

%%% Local Variables:
%%% mode: latex
%%% TeX-PDF-mode: t
%%% TeX-master: "manual"
%%% End:


\chapter{The \why Language}
\label{chap:syntax}

This chapter describes the input syntax, and informally gives its semantics,
illustrated by examples.

A \why text contains a list of \emph{theories}.
A theory is a list of \emph{declarations}. Declarations introduce new
types, functions and predicates, state axioms, lemmas and goals.
These declarations can be directly written in the theory or taken from
existing theories. The base logic of \why is first-order
logic with polymorphic types.

\section{Example 1: Lists}

%BEGIN LATEX
Figure~\ref{fig:tutorial1} contains an example of \why input
text, containing three theories.
%END LATEX
%HEVEA The code below is an example of \why input text, containing three theories.

%BEGIN LATEX
\begin{figure}
\centering
%END LATEX
\begin{whycode}
theory List
  type list 'a = Nil | Cons 'a (list 'a)
end

theory Length
  use import List
  use import int.Int

  function length (l : list 'a) : int =
    match l with
    | Nil      -> 0
    | Cons _ r -> 1 + length r
    end

  lemma Length_nonnegative : forall l:list 'a. length l >= 0
end

theory Sorted
  use import List
  use import int.Int

  inductive sorted (list int) =
    | Sorted_Nil :
        sorted Nil
    | Sorted_One :
        forall x:int. sorted (Cons x Nil)
    | Sorted_Two :
        forall x y : int, l : list int.
        x <= y -> sorted (Cons y l) -> sorted (Cons x (Cons y l))
end
\end{whycode}
%BEGIN LATEX
\vspace*{-1em}%\hrulefill
\caption{Example of \why text}
\label{fig:tutorial1}
\end{figure}
%END LATEX

The first theory, \texttt{List},
declares a new algebraic type for polymorphic lists, \texttt{list 'a}.
As in ML, \texttt{'a} stands for a type variable.
The type \texttt{list 'a} has two constructors, \texttt{Nil} and
\texttt{Cons}. Both constructors can be used as usual function
symbols, respectively of type \texttt{list 'a} and \texttt{'a
  $\times$ list 'a $\rightarrow$ list 'a}.
We deliberately make this theory that short, for reasons which will be
discussed later.

The next theory, \texttt{Length}, introduces the notion of list
length. The \texttt{use import List} command indicates that this new
theory may refer to symbols from theory \texttt{List}. These symbols
are accessible in a qualified form, such as \texttt{List.list} or
\texttt{List.Cons}. The \texttt{import} qualifier additionally allows
us to use them without qualification.

Similarly, the next command \texttt{use import int.Int} adds to our
context the theory \texttt{int.Int} from the standard library. The
prefix \texttt{int} indicates the file in the standard library
containing theory \texttt{Int}. Theories referred to without prefix
either appear earlier in the current file, \eg\ \texttt{List}, or are
predefined.

The next declaration defines a recursive function, \texttt{length},
which computes the length of a list. The \texttt{function} and
\texttt{predicate} keywords are used to introduce function and
predicate symbols, respectively.
\why checks every recursive, or mutually recursive, definition for
termination. Basically, we require a lexicographic and structural
descent for every recursive call for some reordering of arguments.
Notice that matching must be exhaustive and that every \texttt{match}
expression must be terminated by the \texttt{end} keyword.

Despite using higher-order ``curried'' syntax, \why does not permit
partial application: function and predicate arities must be respected.

The last declaration in theory \texttt{Length} is a lemma stating that
the length of a list is non-negative.

The third theory, \texttt{Sorted}, demonstrates the definition of
an inductive predicate. Every such definition is a list of clauses:
universally closed implications where the consequent is an instance
of the defined predicate. Moreover, the defined predicate may only
occur in positive positions in the antecedent. For example, a clause:
\begin{whycode}
  | Sorted_Bad :
      forall x y : int, l : list int.
      (sorted (Cons y l) -> y > x) -> sorted (Cons x (Cons y l))
\end{whycode}
would not be allowed. This positivity condition assures the logical
soundness of an inductive definition.

Note that the type signature of \lstinline{sorted} predicate does not
include the name of a parameter (see \texttt{l} in the definition
of \texttt{length}): it is unused and therefore optional.

\section{Example 1 (continued): Lists and Abstract Orderings}

In the previous section we have seen how a theory can reuse the
declarations of another theory, coming either from the same input
text or from the library. Another way to referring to a theory is
by ``cloning''. A \texttt{clone} declaration constructs a local
copy of the cloned theory, possibly instantiating some of its
abstract (\ie declared but not defined) symbols.

%BEGIN LATEX
Consider the continued example in Figure~\ref{fig:tutorial2}.
%END LATEX
%HEVEA Consider the continued example below.
We write an abstract theory of partial orders, declaring an
abstract type \texttt{t} and an abstract binary predicate
\texttt{<=}. Notice that an infix operation must be enclosed
in parentheses when used outside a term. We also specify
three axioms of a partial order.

%BEGIN LATEX
\begin{figure}
\centering
%END LATEX
\begin{whycode}
theory Order
  type t
  predicate (<=) t t

  axiom Le_refl : forall x : t. x <= x
  axiom Le_asym : forall x y : t. x <= y -> y <= x -> x = y
  axiom Le_trans: forall x y z : t. x <= y -> y <= z -> x <= z
end

theory SortedGen
  use import List
  clone import Order as O

  inductive sorted (l : list t) =
    | Sorted_Nil :
        sorted Nil
    | Sorted_One :
        forall x:t. sorted (Cons x Nil)
    | Sorted_Two :
        forall x y : t, l : list t.
        x <= y -> sorted (Cons y l) -> sorted (Cons x (Cons y l))
end

theory SortedIntList
  use import int.Int
  clone SortedGen with type O.t = int, predicate O.(<=) = (<=)
end
\end{whycode}
%BEGIN LATEX
\vspace*{-1em}%\hrulefill
\caption{Example of \why text (continued)}
\label{fig:tutorial2}
\end{figure}
%END LATEX

There is little value in \texttt{use}'ing such a theory: this
would constrain us to stay with the type \texttt{t}. However,
we can construct an instance of theory \texttt{Order} for
any suitable type and predicate. Moreover, we can build some
further abstract theories using order, and then instantiate
those theories.

Consider theory \texttt{SortedGen}. In the beginning, we
\texttt{use} the earlier theory \texttt{List}. Then we
make a simple \texttt{clone} theory \texttt{Order}.
This is pretty much equivalent to copy-pasting every
declaration from \texttt{Order} to \texttt{SortedGen};
the only difference is that \why traces the history
of cloning and transformations and drivers often make
use of it (see Section~\ref{sec:drivers}).

Notice an important difference between \texttt{use}
and \texttt{clone}. If we \texttt{use} a theory, say
\texttt{List}, twice (directly or indirectly: \eg by
making \texttt{use} of both \texttt{Length} and
\texttt{Sorted}), there is no duplication: there is
still only one type of lists and a unique pair
of constructors. On the contrary, when we \texttt{clone}
a theory, we create a local copy of every cloned
declaration, and the newly created symbols, despite
having the same names, are different from their originals.

Returning to the example, we finish theory \texttt{SortedGen}
with a familiar definition of predicate \texttt{sorted};
this time we use the abstract order on the values of type
\texttt{t}.

Now, we can instantiate theory \texttt{SortedGen} to any
ordered type, without having to retype the definition of
\texttt{sorted}. For example, theory \texttt{SortedIntList}
makes \texttt{clone} of \texttt{SortedGen} (\ie copies its
declarations) substituting type \texttt{int} for type
\texttt{O.t} of \texttt{SortedGen} and the default order
on integers for predicate \texttt{O.(<=)}. \why will
control that the result of cloning is well-typed.

Several remarks ought to be made here. First of all, why should
we clone theory \texttt{Order} in \texttt{SortedGen} if we make
no instantiation? Couldn't we write \texttt{use import Order as O}
instead? The answer is no, we could not. When cloning a theory,
we only can instantiate the symbols declared locally in this theory,
not the symbols imported with \texttt{use}. Therefore, we create
a local copy of \texttt{Order} in \texttt{SortedGen} to be able
to instantiate \texttt{t} and \texttt{(<=)} later.

Secondly, when we instantiate an abstract symbol, its declaration
is not copied from the theory being cloned. Thus, we will not create
a second declaration of type \texttt{int} in \texttt{SortedIntList}.

The mechanism of cloning bears some resemblance to modules and functors
of ML-like languages. Unlike those languages, \why makes no distinction
between modules and module signatures, modules and functors. Any \why
theory can be \texttt{use}'d directly or instantiated in any of its
abstract symbols.

The command-line tool \texttt{why3} (described in
Section~\ref{sec:batch}), allows us to see the effect of cloning.
If the input file containing our example is called \texttt{lists.why},
we can launch the following command:
\begin{verbatim}
> why3 lists.why -T SortedIntList
\end{verbatim}
to see the resulting theory \texttt{SortedIntList}:
\begin{whycode}
theory SortedIntList
  (* use BuiltIn *)
  (* use Int *)
  (* use List *)

  axiom Le_refl : forall x:int. x <= x
  axiom Le_asym : forall x:int, y:int. x <= y -> y <= x -> x = y
  axiom Le_trans : forall x:int, y:int, z:int. x <= y -> y <= z
    -> x <= z

  (* clone Order with type t = int, predicate (<=) = (<=),
    prop Le_trans1 = Le_trans, prop Le_asym1 = Le_asym,
    prop Le_refl1 = Le_refl *)

  inductive sorted (list int) =
    | Sorted_Nil : sorted (Nil:list int)
    | Sorted_One : forall x:int. sorted (Cons x (Nil:list int))
    | Sorted_Two : forall x:int, y:int, l:list int. x <= y ->
        sorted (Cons y l) -> sorted (Cons x (Cons y l))

  (* clone SortedGen with type t1 = int, predicate sorted1 = sorted,
    predicate (<=) = (<=), prop Sorted_Two1 = Sorted_Two,
    prop Sorted_One1 = Sorted_One, prop Sorted_Nil1 = Sorted_Nil,
    prop Le_trans2 = Le_trans, prop Le_asym2 = Le_asym,
    prop Le_refl2 = Le_refl *)
end
\end{whycode}

In conclusion, let us briefly explain the concept of namespaces
in \why. Both \texttt{use} and \texttt{clone} instructions can
be used in three forms (the examples below are given for \texttt{use},
the semantics for \texttt{clone} is the same):
\begin{itemize}
\item \texttt{use List as L} --- every symbol $s$ of theory \texttt{List}
is accessible under the name \texttt{L.$s$}. The \texttt{as L} part is
optional, if it is omitted, the name of the symbol is \texttt{List.$s$}.

\item \texttt{use import List as L} --- every symbol $s$ from
\texttt{List} is accessible under the name \texttt{L.$s$}. It is also
accessible simply as \texttt{$s$}, but only up to the end of the current
namespace, \eg the current theory. If the current theory, that is the
one making \texttt{use}, is later used under the name \texttt{T},
the name of the symbol would be \texttt{T.L.$s$}. (This is why we
could refer directly to the symbols of \texttt{Order} in theory
\texttt{SortedGen}, but had to qualify them with \texttt{O.}
in \texttt{SortedIntList}.)
As in the previous case, \texttt{as L} part is optional.

\item \texttt{use export List} --- every symbol $s$ from \texttt{List}
is accessible simply as \texttt{$s$}. If the current theory
is later used under the name \texttt{T}, the name of the symbol
would be \texttt{T.$s$}.
\end{itemize}

\why allows to open new namespaces explicitly in the text. In particular,
the instruction ``\texttt{clone import Order as O}'' can be equivalently
written as:
\begin{whycode}
namespace import O
  clone export Order
end
\end{whycode}
However, since \why favors short theories over long and complex ones,
this feature is rarely used.

\section{Example 2: Einstein's Problem}
\index{Einstein's logic problem}

We now consider another, slightly more complex example: how to use \why
to solve a little puzzle known as ``Einstein's logic
problem''.%
%BEGIN LATEX
\footnote{This \why example was contributed by St\'ephane Lescuyer.}
%END LATEX
%HEVEA {} (This \why example was contributed by St\'ephane Lescuyer.)
The problem is stated as follows. Five persons, of five
different nationalities, live in five houses in a row, all
painted with different colors.
These five persons own different pets, drink different beverages and
smoke different brands of cigars.
We are given the following information:
\begin{itemize}
\item The Englishman lives in a red house;

\item The Swede has dogs;

\item The Dane drinks tea;

\item The green house is on the left of the white one;

\item The green house's owner drinks coffee;

\item The person who smokes Pall Mall has birds;

\item The yellow house's owner smokes Dunhill;

\item In the house in the center lives someone who drinks milk;

\item The Norwegian lives in the first house;

\item The man who smokes Blends lives next to the one who has cats;

\item The man who owns a horse lives next to the one who smokes Dunhills;

\item The man who smokes Blue Masters drinks beer;

\item The German smokes Prince;

\item The Norwegian lives next to the blue house;

\item The man who smokes Blends has a neighbour who drinks water.
\end{itemize}
The question is: what is the nationality of the fish's owner?

We start by introducing a general-purpose theory defining the notion
of \emph{bijection}, as two abstract types together with two functions from
one to the other and two axioms stating that these functions are
inverse of each other.
\begin{whycode}
theory Bijection
  type t
  type u

  function of t : u
  function to u : t

  axiom To_of : forall x : t. to (of x) = x
  axiom Of_to : forall y : u. of (to y) = y
end
\end{whycode}

We now start a new theory, \texttt{Einstein}, which will contain all
the individuals of the problem.
\begin{whycode}
theory Einstein "Einstein's problem"
\end{whycode}
First we introduce enumeration types for houses, colors, persons,
drinks, cigars and pets.
\begin{whycode}
  type house  = H1 | H2 | H3 | H4 | H5
  type color  = Blue | Green | Red | White | Yellow
  type person = Dane | Englishman | German | Norwegian | Swede
  type drink  = Beer | Coffee | Milk | Tea | Water
  type cigar  = Blend | BlueMaster | Dunhill | PallMall | Prince
  type pet    = Birds | Cats | Dogs | Fish | Horse
\end{whycode}
We now express that each house is associated bijectively to a color,
by cloning the \texttt{Bijection} theory appropriately.
\begin{whycode}
  clone Bijection as Color with type t = house, type u = color
\end{whycode}
It introduces two functions, namely \texttt{Color.of} and
\texttt{Color.to}, from houses to colors and colors to houses,
respectively, and the two axioms relating them.
Similarly, we express that each house is associated bijectively to a
person
\begin{whycode}
  clone Bijection as Owner with type t = house, type u = person
\end{whycode}
and that drinks, cigars and pets are all associated bijectively to persons:
\begin{whycode}
  clone Bijection as Drink with type t = person, type u = drink
  clone Bijection as Cigar with type t = person, type u = cigar
  clone Bijection as Pet   with type t = person, type u = pet
\end{whycode}
Next we need a way to state that a person lives next to another. We
first define a predicate \texttt{leftof} over two houses.
\begin{whycode}
  predicate leftof (h1 h2 : house) =
    match h1, h2 with
    | H1, H2
    | H2, H3
    | H3, H4
    | H4, H5 -> true
    | _      -> false
    end
\end{whycode}
Note how we advantageously used pattern matching, with an or-pattern
for the four positive cases and a universal pattern for the remaining
21 cases. It is then immediate to define a \texttt{neighbour}
predicate over two houses, which completes theory \texttt{Einstein}.
\begin{whycode}
  predicate rightof (h1 h2 : house) =
    leftof h2 h1
  predicate neighbour (h1 h2 : house) =
    leftof h1 h2 \/ rightof h1 h2
end
\end{whycode}

The next theory contains the 15 hypotheses. It starts by importing
theory \texttt{Einstein}.
\begin{whycode}
theory EinsteinHints "Hints"
  use import Einstein
\end{whycode}
Then each hypothesis is stated in terms of \texttt{to} and \texttt{of}
functions. For instance, the hypothesis ``The Englishman lives in a
red house'' is declared as the following axiom.
\begin{whycode}
  axiom Hint1: Color.of (Owner.to Englishman) = Red
\end{whycode}
And so on for all other hypotheses, up to
``The man who smokes Blends has a neighbour who drinks water'', which completes
this theory.
\begin{whycode}
  ...
  axiom Hint15:
    neighbour (Owner.to (Cigar.to Blend)) (Owner.to (Drink.to Water))
end
\end{whycode}
Finally, we declare the goal in the fourth theory:
\begin{whycode}
theory Problem "Goal of Einstein's problem"
  use import Einstein
  use import EinsteinHints

  goal G: Pet.to Fish = German
end
\end{whycode}
and we are ready to use \why to discharge this goal with any prover
of our choice.

%%% Local Variables:
%%% mode: latex
%%% TeX-PDF-mode: t
%%% TeX-master: "manual"
%%% End:


% \input{ide.tex}

\chapter{The \whyml Programming Language}
\label{chap:whyml}

This chapter describes the \whyml programming language.
A \whyml input text contains a list of theories (see
Chapter~\ref{chap:syntax}) and/or modules.
Modules extend theories with \emph{programs}.
Programs can use all types, symbols, and constructs from the logic.
They also provide extra features:
\begin{itemize}
\item
  In a record type declaration, some fields can be declared
  \texttt{mutable} and/or \texttt{ghost}.
\item
  In an algebraic type declaration (this includes record types), an
  invariant can be specified.
\item
  There are programming constructs with no counterpart in the logic:
  \begin{itemize}
  \item mutable field assignment;
  \item sequence;
  \item loops;
  \item exceptions;
  \item local and anonymous functions;
  \item ghost parameters and ghost code;
  \item annotations: pre- and postconditions, assertions, loop invariants.
  \end{itemize}
\item
  A program function can be non-terminating or can be proved
  to be terminating using a variant (a term together with a well-founded
  order relation).
\item
  An abstract program type $t$ can be introduced with a logical
  \emph{model} $\tau$: inside programs, $t$ is abstract, and inside
  annotations, $t$ is an alias for $\tau$.
\end{itemize}
%
Programs are contained in files with suffix \verb|.mlw|.
They are handled by \texttt{why3}. For instance
\begin{verbatim}
> why3 prove myfile.mlw
\end{verbatim}
will display the verification conditions extracted from modules in
file \texttt{myfile.mlw}, as a set of corresponding theories, and
\begin{verbatim}
> why3 prove -P alt-ergo myfile.mlw
\end{verbatim}
will run the SMT solver Alt-Ergo on these verification conditions.
Program files are also handled by the GUI tool \texttt{why3ide}.
See Chapter~\ref{chap:manpages} for more details regarding command lines.

\medskip
As an introduction to \whyml, we use the five problems from the VSTTE
2010 verification competition~\cite{vstte10comp}.
The source code for all these examples is contained in \why's
distribution, in sub-directory \texttt{examples/}.

\section{Problem 1: Sum and Maximum}
\label{sec:MaxAndSum}

The first problem is stated as follows:
\begin{quote}
  Given an $N$-element array of natural numbers,
  write a program to compute the sum and the maximum of the
  elements in the array.
\end{quote}
We  assume $N \ge 0$ and $a[i] \ge 0$ for $0 \le i < N$, as precondition,
and we have to prove the following postcondition:
\begin{displaymath}
  sum \le N \times max.
\end{displaymath}
In a file \verb|max_sum.mlw|, we start a new module:
\begin{whycode}
module MaxAndSum
\end{whycode}
We are obviously needing arithmetic, so we import the corresponding
theory, exactly as we would do within a theory definition:
\begin{whycode}
  use import int.Int
\end{whycode}
We are also going to use references and arrays from \whyml's standard
library, so we import the corresponding modules, with a similar
declaration:
\begin{whycode}
  use import ref.Ref
  use import array.Array
\end{whycode}
Modules \texttt{Ref} and \texttt{Array} respectively provide a type
\texttt{ref 'a} for references and a type \texttt{array 'a} for
arrays, together with useful
operations and traditional syntax. They are loaded from the \whyml
files \texttt{ref.mlw} and \texttt{array.mlw} in the standard library.
\why reports an error when it finds a theory and a module with
the same name in the standard library, or when it finds a theory
declared in a \texttt{.mlw} file and in a \texttt{.why} file with
the same name.

We are now in position to define a program function
\verb|max_sum|. A function definition is introduced with the keyword
\texttt{let}. In our case, it introduces a function with two arguments,
an array \texttt{a} and its size \texttt{n}:
\begin{whycode}
  let max_sum (a: array int) (n: int) = ...
\end{whycode}
(There is a function \texttt{length} to get the size of an array but
we add this extra parameter \texttt{n} to stay close to the original
problem statement.) The function body is a Hoare triple, that is a
precondition, a program expression, and a postcondition.
\begin{whycode}
  let max_sum (a: array int) (n: int)
    requires { 0 <= n = length a }
    requires { forall i:int. 0 <= i < n -> a[i] >= 0 }
    ensures  { let (sum, max) = result in sum <= n * max }
  = ... expression ...
\end{whycode}
The first precondition expresses that \texttt{n} is non-negative and is
equal to the length of \texttt{a} (this will be needed for
verification conditions related to array bound checking).
The second precondition expresses that all
elements of \texttt{a} are non-negative.
The postcondition assumes that the value returned by the function,
denoted \texttt{result}, is a pair of integers, and decomposes it as
the pair \texttt{(sum, max)} to express the required property.
The same postcondition can be written in another form, doing the
pattern matching immediately:
\begin{whycode}
    returns { sum, max -> sum <= n * max }
\end{whycode}

We are now left with the function body itself, that is a code
computing the sum and the maximum of all elements in \texttt{a}. With
no surprise, it is as simple as introducing two local references
\begin{whycode}
    let sum = ref 0 in
    let max = ref 0 in
\end{whycode}
scanning the array with a \texttt{for} loop, updating \texttt{max}
and \texttt{sum}
\begin{whycode}
    for i = 0 to n - 1 do
      if !max < a[i] then max := a[i];
      sum := !sum + a[i]
    done;
\end{whycode}
and finally returning the pair of the values contained in \texttt{sum}
and \texttt{max}:
\begin{whycode}
  (!sum, !max)
\end{whycode}
This completes the code for function \texttt{max\_sum}.
As such, it cannot be proved correct, since the loop is still lacking
a loop invariant. In this case, the loop invariant is as simple as
\verb|!sum <= i * !max|, since the postcondition only requires to prove
\verb|sum <= n * max|. The loop invariant is introduced with the
keyword \texttt{invariant}, immediately after the keyword \texttt{do}.
\begin{whycode}
    for i = 0 to n - 1 do
      invariant { !sum <= i * !max }
      ...
    done
\end{whycode}
There is no need to introduce a variant, as the termination of a
\texttt{for} loop is automatically guaranteed.
This completes module \texttt{MaxAndSum}.
%BEGIN LATEX
Figure~\ref{fig:MaxAndSum} shows the whole code.
\begin{figure}
  \centering
%END LATEX
%HEVEA The whole code is shown below.
\begin{whycode}
module MaxAndSum

  use import int.Int
  use import ref.Ref
  use import array.Array

  let max_sum (a: array int) (n: int)
    requires { 0 <= n = length a }
    requires { forall i:int. 0 <= i < n -> a[i] >= 0 }
    returns  { sum, max -> sum <= n * max }
  = let sum = ref 0 in
    let max = ref 0 in
    for i = 0 to n - 1 do
      invariant { !sum <= i * !max }
      if !max < a[i] then max := a[i];
      sum := !sum + a[i]
    done;
    (!sum, !max)

end
\end{whycode}
%BEGIN LATEX
\vspace*{-1em}%\hrulefill
  \caption{Solution for VSTTE'10 competition problem 1}
  \label{fig:MaxAndSum}
\end{figure}
%END LATEX
We can now proceed to its verification.
Running \texttt{why3}, or better \texttt{why3ide}, on file
\verb|max_sum.mlw| will show a single verification condition with name
\verb|WP_parameter_max_sum|.
Discharging this verification condition with an automated theorem
prover will not succeed, most likely, as it involves non-linear
arithmetic. Repeated applications of goal splitting and calls to
SMT solvers (within \texttt{why3ide}) will typically leave a single,
unsolved goal, which reduces to proving the following sequent:
\begin{displaymath}
  s \le i \times max, ~ max < a[i] \vdash s + a[i] \le (i+1) \times a[i].
\end{displaymath}
This is easily discharged using an interactive proof assistant such as
Coq, and thus completes the verification.

\section{Problem 2: Inverting an Injection}

The second problem is stated as follows:
\begin{quote}
  Invert an injective array $A$ on $N$ elements in the
  subrange from $0$ to $N - 1$, \ie the output array $B$ must be
  such that $B[A[i]] = i$ for $0 \le i < N$.
\end{quote}
We may assume that $A$ is surjective and we have to prove
that the resulting array is also injective.
The code is immediate, since it is as simple as
\begin{whycode}
    for i = 0 to n - 1 do b[a[i]] <- i done
\end{whycode}
so it is more a matter of specification and of getting the proof done
with as much automation as possible. In a new file, we start a new
module and we import arithmetic and arrays:
\begin{whycode}
module InvertingAnInjection
  use import int.Int
  use import array.Array
\end{whycode}
It is convenient to introduce predicate definitions for the properties
of being injective and surjective. These are purely logical
declarations:
\begin{whycode}
  predicate injective (a: array int) (n: int) =
    forall i j: int. 0 <= i < n -> 0 <= j < n -> i <> j -> a[i] <> a[j]

  predicate surjective (a: array int) (n: int) =
    forall i: int. 0 <= i < n -> exists j: int. (0 <= j < n /\ a[j] = i)
\end{whycode}
It is also convenient to introduce the predicate ``being in the
subrange from 0 to $n-1$'':
\begin{whycode}
  predicate range (a: array int) (n: int) =
    forall i: int. 0 <= i < n -> 0 <= a[i] < n
\end{whycode}
Using these predicates, we can formulate the assumption that any
injective array of size $n$ within the range $0..n-1$ is also surjective:
\begin{whycode}
  lemma injective_surjective:
    forall a: array int, n: int.
      injective a n -> range a n -> surjective a n
\end{whycode}
We declare it as a lemma rather than as an axiom, since it is actually
provable. It requires induction and can be proved using the Coq proof
assistant for instance.
Finally we can give the code a specification, with a loop invariant
which simply expresses the values assigned to array \texttt{b} so far:
\begin{whycode}
  let inverting (a: array int) (b: array int) (n: int)
    requires { 0 <= n = length a = length b }
    requires { injective a n /\ range a n }
    ensures  { injective b n }
  = for i = 0 to n - 1 do
      invariant { forall j: int. 0 <= j < i -> b[a[j]] = j }
      b[a[i]] <- i
    done
\end{whycode}
Here we chose to have array \texttt{b} as argument; returning a
freshly allocated array would be equally simple.
%BEGIN LATEX
The whole module is given in Figure~\ref{fig:Inverting}.
%END LATEX
%HEVEA The whole module is given below.
The verification conditions for function \texttt{inverting} are easily
discharged automatically, thanks to the lemma.
%BEGIN LATEX
\begin{figure}
  \centering
%END LATEX
\begin{whycode}
module InvertingAnInjection

  use import int.Int
  use import array.Array

  predicate injective (a: array int) (n: int) =
    forall i j: int. 0 <= i < n -> 0 <= j < n -> i <> j -> a[i] <> a[j]

  predicate surjective (a: array int) (n: int) =
    forall i: int. 0 <= i < n -> exists j: int. (0 <= j < n /\ a[j] = i)

  predicate range (a: array int) (n: int) =
    forall i: int. 0 <= i < n -> 0 <= a[i] < n

  lemma injective_surjective:
    forall a: array int, n: int.
      injective a n -> range a n -> surjective a n

  let inverting (a: array int) (b: array int) (n: int)
    requires { 0 <= n = length a = length b }
    requires { injective a n /\ range a n }
    ensures  { injective b n }
  = for i = 0 to n - 1 do
      invariant { forall j: int. 0 <= j < i -> b[a[j]] = j }
      b[a[i]] <- i
    done

end
\end{whycode}
%BEGIN LATEX
\vspace*{-1em}%\hrulefill
  \caption{Solution for VSTTE'10 competition problem 2}
  \label{fig:Inverting}
\end{figure}
%END LATEX

\section{Problem 3: Searching a Linked List}

The third problem is stated as follows:
\begin{quote}
  Given a linked list representation of a list of integers,
  find the index of the first element that is equal to 0.
\end{quote}
More precisely, the specification says
\begin{quote}
  You have to show that the program returns an index $i$ equal to the
  length of the list if there is no such element. Otherwise, the $i$-th
  element of the list must be equal to 0, and all the preceding
  elements must be non-zero.
\end{quote}
Since the list is not mutated, we can use the algebraic data type of
polymorphic lists from \why's standard library, defined in theory
\texttt{list.List}. It comes with other handy theories:
\texttt{list.Length}, which provides a function \texttt{length}, and
\texttt{list.Nth}, which provides a function \texttt{nth}
for the $n$-th element of a list. The latter returns an option type,
depending on whether the index is meaningful or not.
\begin{whycode}
module SearchingALinkedList
  use import int.Int
  use export list.List
  use export list.Length
  use export list.Nth
\end{whycode}
It is helpful to introduce two predicates: a first one
for a successful search,
\begin{whycode}
  predicate zero_at (l: list int) (i: int) =
    nth i l = Some 0 /\ forall j:int. 0 <= j < i -> nth j l <> Some 0
\end{whycode}
and another for a non-successful search,
\begin{whycode}
  predicate no_zero (l: list int) =
    forall j:int. 0 <= j < length l -> nth j l <> Some 0
\end{whycode}
We are now in position to give the code for the search function.
We write it as a recursive function \texttt{search} that scans a list
for the first zero value:
\begin{whycode}
  let rec search (i: int) (l: list int) =
    match l with
    | Nil      -> i
    | Cons x r -> if x = 0 then i else search (i+1) r
    end
\end{whycode}
Passing an index \texttt{i} as first argument allows to perform a tail
call. A simpler code (yet less efficient) would return 0 in the first
branch and \texttt{1 + search ...} in the second one, avoiding the
extra argument \texttt{i}.

We first prove the termination of this recursive function. It amounts
to give it a \emph{variant}, that is a value that strictly decreases
at each recursive call with respect to some well-founded ordering.
Here it is as simple as the list \texttt{l} itself:
\begin{whycode}
  let rec search (i: int) (l: list int) variant { l } = ...
\end{whycode}
It is worth pointing out that variants are not limited to values
of algebraic types. A non-negative integer term (for example,
\texttt{length l}) can be used, or a term of any other type
equipped with a well-founded order relation.
Several terms can be given, separated with commas,
for lexicographic ordering.

There is no precondition for function \texttt{search}.
The postcondition expresses that either a zero value is found, and
consequently the value returned is bounded accordingly,
\begin{whycode}
  i <= result < i + length l /\ zero_at l (result - i)
\end{whycode}
or no zero value was found, and thus the returned value is exactly
\texttt{i} plus the length of \texttt{l}:
\begin{whycode}
  result = i + length l /\ no_zero l
\end{whycode}
Solving the problem is simply a matter of calling \texttt{search} with
0 as first argument.
%BEGIN LATEX
The code is given Figure~\ref{fig:LinkedList}.
%END LATEX
%HEVEA The code is given below.
The verification conditions are all discharged automatically.
%BEGIN LATEX
\begin{figure}
  \centering
%END LATEX
\begin{whycode}
module SearchingALinkedList

  use import int.Int
  use export list.List
  use export list.Length
  use export list.Nth

  predicate zero_at (l: list int) (i: int) =
    nth i l = Some 0 /\ forall j:int. 0 <= j < i -> nth j l <> Some 0

  predicate no_zero (l: list int) =
    forall j:int. 0 <= j < length l -> nth j l <> Some 0

  let rec search (i: int) (l: list int) variant { l }
    ensures { (i <= result < i + length l /\ zero_at l (result - i))
           \/ (result = i + length l /\ no_zero l) }
  = match l with
    | Nil -> i
    | Cons x r -> if x = 0 then i else search (i+1) r
    end

  let search_list (l: list int)
    ensures { (0 <= result < length l /\ zero_at l result)
           \/ (result = length l /\ no_zero l) }
  = search 0 l

end
\end{whycode}
%BEGIN LATEX
\vspace*{-1em}%\hrulefill
  \caption{Solution for VSTTE'10 competition problem 3}
  \label{fig:LinkedList}
\end{figure}
%END LATEX

Alternatively, we can implement the search with a \texttt{while} loop.
To do this, we need to import references from the standard library,
together with theory \texttt{list.HdTl} which defines functions
\texttt{hd} and \texttt{tl} over lists.
\begin{whycode}
  use import ref.Ref
  use import list.HdTl
\end{whycode}
Being partial functions, \texttt{hd} and \texttt{tl} return options.
For the purpose of our code, though, it is simpler to have functions
which do not return options, but have preconditions instead. Such a
function \texttt{head} is defined as follows:
\begin{whycode}
  let head (l: list 'a)
    requires { l <> Nil } ensures { hd l = Some result }
  = match l with Nil -> absurd | Cons h _ -> h end
\end{whycode}
The program construct \texttt{absurd} denotes an unreachable piece of
code. It generates the verification condition \texttt{false}, which is
here provable using the precondition (the list cannot be \texttt{Nil}).
Function \texttt{tail} is defined similarly:
\begin{whycode}
  let tail (l : list 'a)
    requires { l <> Nil } ensures { tl l = Some result }
  = match l with Nil -> absurd | Cons _ t -> t end
\end{whycode}
Using \texttt{head} and \texttt{tail}, it is straightforward to
implement the search as a \texttt{while} loop.
It uses a local reference \texttt{i} to store the index and another
local reference \texttt{s} to store the list being scanned.
As long as \texttt{s} is not empty and its head is not zero, it
increments \texttt{i} and advances in \texttt{s} using function \texttt{tail}.
\begin{whycode}
  let search_loop l =
    ensures { ... same postcondition as in search_list ... }
  = let i = ref 0 in
    let s = ref l in
    while !s <> Nil && head !s <> 0 do
      invariant { ... }
      variant   { !s }
      i := !i + 1;
      s := tail !s
    done;
    !i
\end{whycode}
The postcondition is exactly the same as for function \verb|search_list|.
The termination of the \texttt{while} loop is ensured using a variant,
exactly as for a recursive function. Such a variant must strictly decrease at
each execution of the loop body. The reader is invited to figure out
the loop invariant.

\section{Problem 4: N-Queens}

The fourth problem is probably the most challenging one.
We have to verify the implementation of a program which solves the
$N$-queens puzzle: place $N$ queens on an $N \times N$
chess board so that no queen can capture another one with a
legal move.
The program should return a placement if there is a solution and
indicates that there is no solution otherwise. A placement is a
$N$-element array which assigns the queen on row $i$ to its column.
Thus we start our module by importing arithmetic and arrays:
\begin{whycode}
module NQueens
  use import int.Int
  use import array.Array
\end{whycode}
The code is a simple backtracking algorithm, which tries to put a queen
on each row of the chess board, one by one (there is basically no
better way to solve the $N$-queens puzzle).
A building block is a function which checks whether the queen on a
given row may attack another queen on a previous row. To verify this
function, we first define a more elementary predicate, which expresses
that queens on row \texttt{pos} and \texttt{q} do no attack each other:
\begin{whycode}
  predicate consistent_row (board: array int) (pos: int) (q: int) =
    board[q] <> board[pos] /\
    board[q] - board[pos] <> pos - q /\
    board[pos] - board[q] <> pos - q
\end{whycode}
Then it is possible to define the consistency of row \texttt{pos}
with respect to all previous rows:
\begin{whycode}
  predicate is_consistent (board: array int) (pos: int) =
    forall q:int. 0 <= q < pos -> consistent_row board pos q
\end{whycode}
Implementing a function which decides this predicate is another
matter. In order for it to be efficient, we want to return
\texttt{False} as soon as a queen attacks the queen on row
\texttt{pos}. We use an exception for this purpose and it carries the
row of the attacking queen:
\begin{whycode}
  exception Inconsistent int
\end{whycode}
The check is implemented by a function \verb|check_is_consistent|,
which takes the board and the row \texttt{pos} as arguments, and scans
rows from 0 to \texttt{pos-1} looking for an attacking queen. As soon
as one is found, the exception is raised. It is caught immediately
outside the loop and \texttt{False} is returned. Whenever the end of
the loop is reached, \texttt{True} is returned.
\begin{whycode}
  let check_is_consistent (board: array int) (pos: int)
    requires { 0 <= pos < length board }
    ensures  { result=True <-> is_consistent board pos }
  = try
      for q = 0 to pos - 1 do
        invariant {
          forall j:int. 0 <= j < q -> consistent_row board pos j
        }
        let bq   = board[q]   in
        let bpos = board[pos] in
        if bq        = bpos    then raise (Inconsistent q);
        if bq - bpos = pos - q then raise (Inconsistent q);
        if bpos - bq = pos - q then raise (Inconsistent q)
      done;
      True
    with Inconsistent q ->
      assert { not (consistent_row board pos q) };
      False
    end
\end{whycode}
The assertion in the exception handler is a cut for SMT solvers.
%BEGIN LATEX
This first part of the solution is given in Figure~\ref{fig:NQueens1}.
%END LATEX
%HEVEA This first part of the solution is given below.
%BEGIN LATEX
\begin{figure}
  \centering
%END LATEX
\begin{whycode}
module NQueens
  use import int.Int
  use import array.Array

  predicate consistent_row (board: array int) (pos: int) (q: int) =
    board[q] <> board[pos] /\
    board[q] - board[pos] <> pos - q /\
    board[pos] - board[q] <> pos - q

  predicate is_consistent (board: array int) (pos: int) =
    forall q:int. 0 <= q < pos -> consistent_row board pos q

  exception Inconsistent int

  let check_is_consistent (board: array int) (pos: int)
    requires { 0 <= pos < length board }
    ensures  { result=True <-> is_consistent board pos }
  = try
      for q = 0 to pos - 1 do
        invariant {
          forall j:int. 0 <= j < q -> consistent_row board pos j
        }
        let bq   = board[q]   in
        let bpos = board[pos] in
        if bq        = bpos    then raise (Inconsistent q);
        if bq - bpos = pos - q then raise (Inconsistent q);
        if bpos - bq = pos - q then raise (Inconsistent q)
      done;
      True
    with Inconsistent q ->
      assert { not (consistent_row board pos q) };
      False
    end
\end{whycode}
%BEGIN LATEX
\vspace*{-1em}%\hrulefill
  \caption{Solution for VSTTE'10 competition problem 4 (1/2)}
  \label{fig:NQueens1}
\end{figure}
%END LATEX

We now proceed with the verification of the backtracking algorithm.
The specification requires us to define the notion of solution, which
is straightforward using the predicate \verb|is_consistent| above.
However, since the algorithm will try to complete a given partial
solution, it is more convenient to define the notion of partial
solution, up to a given row. It is even more convenient to split it in
two predicates, one related to legal column values and another to
consistency of rows:
\begin{whycode}
  predicate is_board (board: array int) (pos: int) =
    forall q:int. 0 <= q < pos -> 0 <= board[q] < length board

  predicate solution (board: array int) (pos: int) =
    is_board board pos /\
    forall q:int. 0 <= q < pos -> is_consistent board q
\end{whycode}
The algorithm will not mutate the partial solution it is given and,
in case of a search failure, will claim that there is no solution
extending this prefix. For this reason, we introduce a predicate
comparing two chess boards for equality up to a given row:
\begin{whycode}
  predicate eq_board (b1 b2: array int) (pos: int) =
    forall q:int. 0 <= q < pos -> b1[q] = b2[q]
\end{whycode}
The search itself makes use of an exception to signal a successful search:
\begin{whycode}
  exception Solution
\end{whycode}
The backtracking code is a recursive function \verb|bt_queens| which
takes the chess board, its size, and the starting row for the search.
The termination is ensured by the obvious variant \texttt{n-pos}.
\begin{whycode}
  let rec bt_queens (board: array int) (n: int) (pos: int)
    variant  { n-pos }
\end{whycode}
The precondition relates \texttt{board}, \texttt{pos}, and \texttt{n}
and requires \texttt{board} to be a solution up to \texttt{pos}:
\begin{whycode}
    requires { 0 <= pos <= n = length board }
    requires { solution board pos }
\end{whycode}
The postcondition is twofold: either the function exits normally and
then there is no solution extending the prefix in \texttt{board},
which has not been modified;
or the function raises \texttt{Solution} and we have a solution in
\texttt{board}.
\begin{whycode}
    ensures  { eq_board board (old board) pos }
    ensures  { forall b:array int. length b = n -> is_board b n ->
                 eq_board board b pos -> not (solution b n) }
    raises   { Solution -> solution board n }
  = 'Init:
\end{whycode}
We place a code mark \texttt{'Init} immediately at the beginning of
the program body to
be able to refer to the value of \texttt{board} in the pre-state.
Whenever we reach the end of the chess board, we have found a solution
and we signal it using exception \texttt{Solution}:
\begin{whycode}
    if pos = n then raise Solution;
\end{whycode}
Otherwise we scan all possible positions for the queen on row
\texttt{pos} with a \texttt{for} loop:
\begin{whycode}
    for i = 0 to n - 1 do
\end{whycode}
The loop invariant states that we have not modified the solution
prefix so far, and that we have not found any solution that would
extend this prefix with a queen on row \texttt{pos} at a column below
\texttt{i}:
\begin{whycode}
      invariant { eq_board board (at board 'Init) pos }
      invariant { forall b:array int.  length b = n -> is_board b n ->
        eq_board board b pos -> 0 <= b[pos] < i -> not (solution b n) }
\end{whycode}
Then we assign column \texttt{i} to the queen on row \texttt{pos} and
we check for a possible attack with \verb|check_is_consistent|. If
not, we call \verb|bt_queens| recursively on the next row.
\begin{whycode}
      board[pos] <- i;
      if check_is_consistent board pos then bt_queens board n (pos + 1)
    done
\end{whycode}
This completes the loop and function \verb|bt_queens| as well.
Solving the puzzle is a simple call to \verb|bt_queens|, starting the
search on row 0. The postcondition is also twofold, as for
\verb|bt_queens|, yet slightly simpler.
\begin{whycode}
  let queens (board: array int) (n: int)
    requires { 0 <= length board = n }
    ensures  { forall b:array int.
                 length b = n -> is_board b n -> not (solution b n) }
    raises   { Solution -> solution board n }
  = bt_queens board n 0
\end{whycode}
%BEGIN LATEX
This second part of the solution is given Figure~\ref{fig:NQueens2}.
%END LATEX
%HEVEA This second part of the solution is given below.
With the help of a few auxiliary lemmas --- not given here but available
from \why's sources --- the verification conditions are all discharged
automatically, including the verification of the lemmas themselves.
%BEGIN LATEX
\begin{figure}
  \centering
%END LATEX
\begin{whycode}
  predicate is_board (board: array int) (pos: int) =
    forall q:int. 0 <= q < pos -> 0 <= board[q] < length board

  predicate solution (board: array int) (pos: int) =
    is_board board pos /\
    forall q:int. 0 <= q < pos -> is_consistent board q

  predicate eq_board (b1 b2: array int) (pos: int) =
    forall q:int. 0 <= q < pos -> b1[q] = b2[q]

  exception Solution

  let rec bt_queens (board: array int) (n: int) (pos: int)
    variant  { n - pos }
    requires { 0 <= pos <= n = length board }
    requires { solution board pos }
    ensures  { eq_board board (old board) pos }
    ensures  { forall b:array int. length b = n -> is_board b n ->
                 eq_board board b pos -> not (solution b n) }
    raises   { Solution -> solution board n }
  = 'Init:
    if pos = n then raise Solution;
    for i = 0 to n - 1 do
      invariant { eq_board board (at board 'Init) pos }
      invariant { forall b:array int. length b = n -> is_board b n ->
        eq_board board b pos -> 0 <= b[pos] < i -> not (solution b n) }
      board[pos] <- i;
      if check_is_consistent board pos then bt_queens board n (pos + 1)
    done

  let queens (board: array int) (n: int)
    requires { 0 <= length board = n }
    ensures  { forall b:array int.
                 length b = n -> is_board b n -> not (solution b n) }
    raises   { Solution -> solution board n }
  = bt_queens board n 0

end
\end{whycode}
%BEGIN LATEX
\vspace*{-1em}%\hrulefill
  \caption{Solution for VSTTE'10 competition problem 4 (2/2)}
  \label{fig:NQueens2}
\end{figure}
%END LATEX

\section{Problem 5: Amortized Queue}

The last problem consists in verifying the implementation of a
well-known purely applicative data structure for queues.
A queue is composed of two lists, \textit{front} and \textit{rear}.
We push elements at the head of list \textit{rear} and pop them off
the head of list \textit{front}. We maintain that the length of
\textit{front} is always greater or equal to the length of \textit{rear}.
(See for instance Okasaki's \emph{Purely Functional Data
  Structures}~\cite{okasaki98} for more details.)

We have to implement operations \texttt{empty}, \texttt{head},
\texttt{tail}, and \texttt{enqueue} over this data type,
to show that the invariant over lengths is maintained, and finally
\begin{quote}
  to show that a client invoking these operations
  observes an abstract queue given by a sequence.
\end{quote}
In a new module, we import arithmetic and theory
\texttt{list.ListRich}, a combo theory that imports all list
operations we will require: length, reversal, and concatenation.
\begin{whycode}
module AmortizedQueue
  use import int.Int
  use export list.ListRich
\end{whycode}
The queue data type is naturally introduced as a polymorphic record type.
The two list lengths are explicitly stored, for better efficiency.
\begin{whycode}
  type queue 'a = { front: list 'a; lenf: int;
                    rear : list 'a; lenr: int; }
  invariant {
    length self.front = self.lenf >= length self.rear = self.lenr }
\end{whycode}
The type definition is accompanied with an invariant ---
a logical property imposed on any value of the type.
\why assumes that any \texttt{queue} passed as an argument to
a program function satisfies the invariant and it produces
a proof obligation every time a \texttt{queue} is created
or modified in a program.

For the purpose of the specification, it is convenient to introduce a function
\texttt{sequence} which builds the sequence of elements of a queue, that
is the front list concatenated to the reversed rear list.
\begin{whycode}
  function sequence (q: queue 'a) : list 'a = q.front ++ reverse q.rear
\end{whycode}
It is worth pointing out that this function will only be used in
specifications.
We start with the easiest operation: building the empty queue.
\begin{whycode}
  let empty () ensures { sequence result = Nil }
  = { front = Nil; lenf = 0; rear = Nil; lenr = 0 } : queue 'a
\end{whycode}
The postcondition states that the returned queue represents
the empty sequence. Another postcondition, saying that the
returned queue satisfies the type invariant, is implicit.
Note the cast to type \texttt{queue 'a}. It is required, for the
type checker not to complain about an undefined type variable.

The next operation is \texttt{head}, which returns the first element from
a given queue \texttt{q}. It naturally requires the queue to be non
empty, which is conveniently expressed as \texttt{sequence q} not
being \texttt{Nil}.
\begin{whycode}
  let head (q: queue 'a)
    requires { sequence q <> Nil }
    ensures { hd (sequence q) = Some result }
  = match q.front with
      | Nil      -> absurd
      | Cons x _ -> x
    end
\end{whycode}
That the argument \texttt{q} satisfies the type invariant is
implicitly assumed. The type invariant is
required to prove the absurdity of the first branch (if
\texttt{q.front} is \texttt{Nil}, then so should be \texttt{sequence q}).

The next operation is \texttt{tail}, which removes the first element
from a given queue. This is more subtle than \texttt{head}, since we
may have to re-structure the queue to maintain the invariant.
Since we will have to perform a similar operation when implementation
operation \texttt{enqueue}, it is a good idea to introduce a smart
constructor \texttt{create} which builds a queue from two lists, while
ensuring the invariant. The list lengths are also passed as arguments,
to avoid unnecessary computations.
\begin{whycode}
  let create (f: list 'a) (lf: int) (r: list 'a) (lr: int)
    requires { lf = length f /\ lr = length r }
    ensures  { sequence result = f ++ reverse r }
  = if lf >= lr then
      { front = f; lenf = lf; rear = r; lenr = lr }
    else
      let f = f ++ reverse r in
      { front = f; lenf = lf + lr; rear = Nil; lenr = 0 }
\end{whycode}
If the invariant already holds, it is simply a matter of building the
record. Otherwise, we empty the rear list and build a new front list
as the concatenation of list \texttt{f} and the reversal of list \texttt{r}.
The principle of this implementation is that the cost of this reversal
will be amortized over all queue operations. Implementing function
\texttt{tail} is now straightforward and follows the structure of
function \texttt{head}.
\begin{whycode}
  let tail (q: queue 'a)
    requires { sequence q <> Nil }
    ensures  { tl (sequence q) = Some (sequence result) }
  = match q.front with
      | Nil      -> absurd
      | Cons _ r -> create r (q.lenf - 1) q.rear q.lenr
    end
\end{whycode}
The last operation is \texttt{enqueue}, which pushes a new element in
a given queue. Reusing the smart constructor \texttt{create} makes it
a one line code.
\begin{whycode}
  let enqueue (x: 'a) (q: queue 'a)
    ensures { sequence result = sequence q ++ Cons x Nil }
  = create q.front q.lenf (Cons x q.rear) (q.lenr + 1)
\end{whycode}
%BEGIN LATEX
The code is given Figure~\ref{fig:AQueue}.
%END LATEX
%HEVEA The code is given below.
The verification conditions are all discharged automatically.
%BEGIN LATEX
\begin{figure}[p]
  \centering
%END LATEX
\begin{whycode}
module AmortizedQueue
  use import int.Int
  use export list.ListRich

  type queue 'a = { front: list 'a; lenf: int;
                    rear : list 'a; lenr: int; }
  invariant {
    length self.front = self.lenf >= length self.rear = self.lenr }

  function sequence (q: queue 'a) : list 'a = q.front ++ reverse q.rear

  let empty () ensures { sequence result = Nil }
  = { front = Nil; lenf = 0; rear = Nil; lenr = 0 } : queue 'a

  let head (q: queue 'a)
    requires { sequence q <> Nil }
    ensures { hd (sequence q) = Some result }
  = match q.front with
      | Nil      -> absurd
      | Cons x _ -> x
    end

  let create (f: list 'a) (lf: int) (r: list 'a) (lr: int)
    requires { lf = length f /\ lr = length r }
    ensures  { sequence result = f ++ reverse r }
  = if lf >= lr then
      { front = f; lenf = lf; rear = r; lenr = lr }
    else
      let f = f ++ reverse r in
      { front = f; lenf = lf + lr; rear = Nil; lenr = 0 }

  let tail (q: queue 'a)
    requires { sequence q <> Nil }
    ensures  { tl (sequence q) = Some (sequence result) }
  = match q.front with
      | Nil      -> absurd
      | Cons _ r -> create r (q.lenf - 1) q.rear q.lenr
    end

  let enqueue (x: 'a) (q: queue 'a)
    ensures { sequence result = sequence q ++ Cons x Nil }
  = create q.front q.lenf (Cons x q.rear) (q.lenr + 1)
end
\end{whycode}
%BEGIN LATEX
\vspace*{-1em}%\hrulefill
  \caption{Solution for VSTTE'10 competition problem 5}
  \label{fig:AQueue}
\end{figure}
%END LATEX

% other examples: same fringe ?

%%% Local Variables:
%%% compile-command: "make -C .. doc"
%%% mode: latex
%%% TeX-PDF-mode: t
%%% TeX-master: "manual"
%%% End:

% LocalWords:  surjective


\chapter{The \why API}
\label{chap:api}\index{API}

This chapter is a tutorial for the users who want to link their own
OCaml code with the \why library. We progressively introduce the way
one can use the library to build terms, formulas, theories, proof
tasks, call external provers on tasks, and apply transformations on
tasks. The complete documentation for API calls is given
at URL~\url{http://why3.lri.fr/api-\whyversion/}.

We assume the reader has a fair knowledge of the OCaml
language. Notice that the \why library must be installed, see
Section~\ref{sec:installlib}. The OCaml code given below is available in
the source distribution in directory \verb|examples/use_api/| together
with a few other examples.


\section{Building Propositional Formulas}

The first step is to know how to build propositional formulas. The
module \texttt{Term} gives a few functions for building these. Here is
a piece of OCaml code for building the formula $\mathit{true} \lor
\mathit{false}$.
\begin{ocamlcode}
(* opening the Why3 library *)
open Why3

(* a ground propositional goal: true or false *)
let fmla_true : Term.term = Term.t_true
let fmla_false : Term.term = Term.t_false
let fmla1 : Term.term = Term.t_or fmla_true fmla_false
\end{ocamlcode}
The library uses the common type \texttt{term} both for terms
(\ie expressions that produce a value of some particular type)
and formulas (\ie boolean-valued expressions).
% To distinguish terms from formulas, one can look at the
% \texttt{t_ty} field of the \texttt{term} record: in formulas,
% this field has the value \texttt{None}, and in terms,
% \texttt{Some t}, where \texttt{t} is of type \texttt{Ty.ty}.

Such a formula can be printed using the module \texttt{Pretty}
providing pretty-printers.
\begin{ocamlcode}
(* printing it *)
open Format
let () = printf "@[formula 1 is:@ %a@]@." Pretty.print_term fmla1
\end{ocamlcode}

Assuming the lines above are written in a file \texttt{f.ml}, it can
be compiled using
\begin{verbatim}
ocamlc str.cma unix.cma nums.cma dynlink.cma \
        -I +ocamlgraph -I +why3 graph.cma why.cma f.ml -o f
\end{verbatim}
Running the generated executable \texttt{f} results in the following output.
\begin{verbatim}
formula 1 is: true \/ false
\end{verbatim}

Let us now build a formula with propositional variables: $A \land B
\rightarrow A$. Propositional variables must be declared first before
using them in formulas. This is done as follows.
\begin{ocamlcode}
let prop_var_A : Term.lsymbol =
  Term.create_psymbol (Ident.id_fresh "A") []
let prop_var_B : Term.lsymbol =
  Term.create_psymbol (Ident.id_fresh "B") []
\end{ocamlcode}
The type \texttt{lsymbol} is the type of function and predicate symbols (which
we call logic symbols for brevity). Then the atoms $A$ and $B$ must be built
by the general function for applying a predicate symbol to a list of terms.
Here we just need the empty list of arguments.
\begin{ocamlcode}
let atom_A : Term.term = Term.ps_app prop_var_A []
let atom_B : Term.term = Term.ps_app prop_var_B []
let fmla2 : Term.term =
  Term.t_implies (Term.t_and atom_A atom_B) atom_A
let () = printf "@[formula 2 is:@ %a@]@." Pretty.print_term fmla2
\end{ocamlcode}

As expected, the output is as follows.
\begin{verbatim}
formula 2 is: A /\ B -> A
\end{verbatim}
Notice that the concrete syntax of \why forbids function and predicate
names to start with a capital letter (except for the algebraic type
constructors which must start with one). This constraint is not enforced
when building those directly using library calls.

\section{Building Tasks}

Let us see how we can call a prover to prove a formula. As said in
previous chapters, a prover must be given a task, so we need to build
tasks from our formulas. Task can be build incrementally from an empty
task by adding declaration to it, using the functions
\texttt{add\_*\_decl} of module \texttt{Task}. For the formula $\mathit{true} \lor
\mathit{false}$ above, this is done as follows.
\begin{ocamlcode}
let task1 : Task.task = None (* empty task *)
let goal_id1 : Decl.prsymbol =
  Decl.create_prsymbol (Ident.id_fresh "goal1")
let task1 : Task.task =
  Task.add_prop_decl task1 Decl.Pgoal goal_id1 fmla1
\end{ocamlcode}
To make the formula a goal, we must give a name to it, here ``goal1''. A
goal name has type \texttt{prsymbol}, for identifiers denoting
propositions in a theory or a task. Notice again that the concrete
syntax of \why requires these symbols to be capitalized, but it is not
mandatory when using the library. The second argument of
\texttt{add\_prop\_decl} is the kind of the proposition:
\texttt{Paxiom}, \texttt{Plemma} or \texttt{Pgoal}.
Notice that lemmas are not allowed in tasks
and can only be used in theories.

Once a task is built, it can be printed.
\begin{ocamlcode}
(* printing the task *)
let () = printf "@[task 1 is:@\n%a@]@." Pretty.print_task task1
\end{ocamlcode}

The task for our second formula is a bit more complex to build, because
the variables A and B must be added as abstract (\ie not defined)
propositional symbols in the task.
\begin{ocamlcode}
(* task for formula 2 *)
let task2 = None
let task2 = Task.add_param_decl task2 prop_var_A
let task2 = Task.add_param_decl task2 prop_var_B
let goal_id2 = Decl.create_prsymbol (Ident.id_fresh "goal2")
let task2 = Task.add_prop_decl task2 Decl.Pgoal goal_id2 fmla2
let () = printf "@[task 2 is:@\n%a@]@." Pretty.print_task task2
\end{ocamlcode}

Execution of our OCaml program now outputs:
\begin{verbatim}
task 1 is:
theory Task
  goal Goal1 : true \/ false
end

task 2 is:
theory Task
  predicate A

  predicate B

  goal Goal2 : A /\ B -> A
end
\end{verbatim}

\section{Calling External Provers}

To call an external prover, we need to access the \why configuration
file \texttt{why3.conf}, as it was built using the \texttt{why3config}
command line tool or the \textsf{Detect Provers} menu of the graphical
IDE. The following API calls allow to access the content of this
configuration file.
\begin{ocamlcode}
(* reads the config file *)
let config : Whyconf.config = Whyconf.read_config None
(* the [main] section of the config file *)
let main : Whyconf.main = Whyconf.get_main config
(* all the provers detected, from the config file *)
let provers : Whyconf.config_prover Whyconf.Mprover.t =
  Whyconf.get_provers config
\end{ocamlcode}
The type \texttt{'a Whyconf.Mprover.t} is a map indexed by provers. A
prover is a record with a name, a version, and an alternative description
(to differentiate between various configurations of a given prover). Its
definition is in the module \texttt{Whyconf}:
\begin{ocamlcode}
type prover =
    { prover_name : string; (* "Alt-Ergo" *)
      prover_version : string; (* "2.95" *)
      prover_altern : string; (* "special" *)
    }
\end{ocamlcode}
The map \texttt{provers} provides the set of existing provers.
In the following, we directly
attempt to access the prover Alt-Ergo, which is known to be identified
with id \texttt{"alt-ergo"}.
\begin{ocamlcode}
(* the [prover alt-ergo] section of the config file *)
let alt_ergo : Whyconf.config_prover =
  try
    Whyconf.prover_by_id config "alt-ergo"
  with Whyconf.ProverNotFound _ ->
    eprintf "Prover alt-ergo not installed or not configured@.";
    exit 0
\end{ocamlcode}
We could also get a specific version with :
\begin{ocamlcode}
let alt_ergo : Whyconf.config_prover =
  try
    let prover = {Whyconf.prover_name = "Alt-Ergo";
                  prover_version = "0.92.3";
                  prover_altern = ""} in
    Whyconf.Mprover.find prover provers
  with Not_found ->
    eprintf "Prover alt-ergo not installed or not configured@.";
    exit 0
\end{ocamlcode}

The next step is to obtain the driver associated to this prover. A
driver typically depends on the standard theories so these should be
loaded first.
\begin{ocamlcode}
(* builds the environment from the [loadpath] *)
let env : Env.env =
  Env.create_env (Whyconf.loadpath main)
(* loading the Alt-Ergo driver *)
let alt_ergo_driver : Driver.driver =
  try
    Driver.load_driver env alt_ergo.Whyconf.driver
  with e ->
    eprintf "Failed to load driver for alt-ergo: %a@."
      Exn_printer.exn_printer e;
    exit 1
\end{ocamlcode}

We are now ready to call the prover on the tasks. This is done by a
function call that launches the external executable and waits for its
termination. Here is a simple way to proceed:
\begin{ocamlcode}
(* calls Alt-Ergo *)
let result1 : Call_provers.prover_result =
  Call_provers.wait_on_call
    (Driver.prove_task ~command:alt_ergo.Whyconf.command
    alt_ergo_driver task1 ()) ()
(* prints Alt-Ergo answer *)
let () = printf "@[On task 1, alt-ergo answers %a@]@."
  Call_provers.print_prover_result result1
\end{ocamlcode}
This way to call a prover is in general too naive, since it may never
return if the prover runs without time limit. The function
\texttt{prove\_task} has two optional parameters: \texttt{timelimit}
is the maximum allowed running time in seconds, and \texttt{memlimit}
is the maximum allowed memory in megabytes.  The type
\texttt{prover\_result} is a record with three fields:
\begin{itemize}
\item \texttt{pr\_answer}: the prover answer, explained below;
\item \texttt{pr\_output}: the output of the prover, \ie both
  standard output and the standard error of the process
  (a redirection in \texttt{why3.conf} is required);
\item \texttt{pr\_time} : the time taken by the prover, in seconds.
\end{itemize}
A \texttt{pr\_answer} is a sum of several kind of answers:
\begin{itemize}
\item \texttt{Valid}: the task is valid according to the prover.
\item \texttt{Invalid}: the task is invalid.
\item \texttt{Timeout}: the prover exceeds the time or memory limit.
\item \texttt{Unknown} $msg$: the prover can't determine if the task
  is valid; the string parameter $msg$ indicates some extra
  information.
\item \texttt{Failure} $msg$: the prover reports a failure, \ie it
  was unable to read correctly its input task.
\item \texttt{HighFailure}: an error occurred while trying to call the
  prover, or the prover answer was not understood (\ie none of the
  given regular expressions in the driver file matches the output
  of the prover).
\end{itemize}
Here is thus another way of calling the Alt-Ergo prover, on our second
task.
\begin{ocamlcode}
let result2 : Call_provers.prover_result =
   Call_provers.wait_on_call
    (Driver.prove_task ~command:alt_ergo.Whyconf.command
    ~timelimit:10
    alt_ergo_driver task2 ()) ()

let () =
  printf "@[On task 2, alt-ergo answers %a in %5.2f seconds@."
    Call_provers.print_prover_answer
    result1.Call_provers.pr_answer
    result1.Call_provers.pr_time
\end{ocamlcode}
The output of our program is now as follows.
\begin{verbatim}
On task 1, alt-ergo answers Valid (0.01s)
On task 2, alt-ergo answers Valid in  0.01 seconds
\end{verbatim}

\section{Building Terms}

An important feature of the functions for building terms and formulas
is that they statically guarantee that only well-typed terms can be
constructed.

Here is the way we build the formula $2+2=4$. The main difficulty is to
access the internal identifier for addition: it must be retrieved from
the standard theory \texttt{Int} of the file \texttt{int.why} (see
Chap~\ref{sec:library}).
\begin{ocamlcode}
let two : Term.term = 
  Term.t_const (Number.ConstInt (Number.int_const_dec "2"))
let four : Term.term = 
  Term.t_const (Number.ConstInt (Number.int_const_dec "4"))
let int_theory : Theory.theory =
  Env.read_theory env ["int"] "Int"
let plus_symbol : Term.lsymbol =
  Theory.ns_find_ls int_theory.Theory.th_export ["infix +"]
let two_plus_two : Term.term =
  Term.t_app_infer plus_symbol [two;two]
let fmla3 : Term.term = Term.t_equ two_plus_two four
\end{ocamlcode}
An important point to notice as that when building the application of
$+$ to the arguments, it is checked that the types are correct. Indeed
the constructor \texttt{t\_app\_infer} infers the type of the resulting
term. One could also provide the expected type as follows.
\begin{ocamlcode}
let two_plus_two : Term.term =
  Term.fs_app plus_symbol [two;two] Ty.ty_int
\end{ocamlcode}

When building a task with this formula, we need to declare that we use
theory \texttt{Int}:
\begin{ocamlcode}
let task3 = None
let task3 = Task.use_export task3 int_theory
let goal_id3 = Decl.create_prsymbol (Ident.id_fresh "goal3")
let task3 = Task.add_prop_decl task3 Decl.Pgoal goal_id3 fmla3
\end{ocamlcode}

\section{Building Quantified Formulas}

To illustrate how to build quantified formulas, let us consider
the formula $\forall x:int. x*x \geq 0$. The first step is to
obtain the symbols from \texttt{Int}.
\begin{ocamlcode}
let zero : Term.term = 
  Term.t_const (Number.ConstInt (Number.int_const_dec "0"))
let mult_symbol : Term.lsymbol =
  Theory.ns_find_ls int_theory.Theory.th_export ["infix *"]
let ge_symbol : Term.lsymbol =
  Theory.ns_find_ls int_theory.Theory.th_export ["infix >="]
\end{ocamlcode}
The next step is to introduce the variable $x$ with the type int.
\begin{ocamlcode}
let var_x : Term.vsymbol =
  Term.create_vsymbol (Ident.id_fresh "x") Ty.ty_int
\end{ocamlcode}
The formula $x*x \geq 0$ is obtained as in the previous example.
\begin{ocamlcode}
let x : Term.term = Term.t_var var_x
let x_times_x : Term.term = Term.t_app_infer mult_symbol [x;x]
let fmla4_aux : Term.term = Term.ps_app ge_symbol [x_times_x;zero]
\end{ocamlcode}
To quantify on $x$, we use the appropriate smart constructor as follows.
\begin{ocamlcode}
let fmla4 : Term.term = Term.t_forall_close [var_x] [] fmla4_aux
\end{ocamlcode}

\section{Building Theories}

We illustrate now how one can build theories. Building a theory must
be done by a sequence of calls:
\begin{itemize}
\item creating a theory ``under construction'', of type \verb|Theory.theory_uc|;
\item adding declarations, one at a time;
\item closing the theory under construction, obtaining something of type \verb|Theory.theory|.
\end{itemize}

Creation of a theory named \verb|My_theory| is done by
\begin{ocamlcode}
let my_theory : Theory.theory_uc = 
  Theory.create_theory (Ident.id_fresh "My_theory")
\end{ocamlcode}

First let us add formula 1 above as a goal:
\begin{ocamlcode}
let decl_goal1 : Decl.decl =
  Decl.create_prop_decl Decl.Pgoal goal_id1 fmla1 
let my_theory : Theory.theory_uc =
  Theory.add_decl my_theory decl_goal1
\end{ocamlcode}
Note that we reused the goal identifier \verb|goal_id1| that we
already defined to create task 1 above.

Adding formula 2 needs to add the declarations of predicate variables A
and B first:
\begin{ocamlcode}
let my_theory : Theory.theory_uc = 
  Theory.add_param_decl my_theory prop_var_A 
let my_theory : Theory.theory_uc = 
  Theory.add_param_decl my_theory prop_var_B 
let decl_goal2 : Decl.decl =
  Decl.create_prop_decl Decl.Pgoal goal_id2 fmla2 
let my_theory : Theory.theory_uc = Theory.add_decl my_theory decl_goal2
\end{ocamlcode}

Adding formula 3 is a bit more complex since it uses integers, thus it
requires to ``use'' the theory \verb|int.Int|. Using a theory is
indeed not a primitive operation in the API: it must be done by a
combination of an ``export'' and the creation of a namespace. We
provide a helper function for that:
\begin{ocamlcode}
(* [use th1 th2] insert the equivalent of a "use import th2" in
  theory th1 under construction *) 
let use th1 th2 = 
  let name = th2.Theory.th_name in 
  Theory.close_namespace 
    (Theory.use_export 
      (Theory.open_namespace th1 name.Ident.id_string) th2) true
\end{ocamlcode}
Addition of formula 3 is then
\begin{ocamlcode}
let my_theory : Theory.theory_uc = use my_theory int_theory
let decl_goal3 : Decl.decl = 
  Decl.create_prop_decl Decl.Pgoal goal_id3 fmla3
let my_theory : Theory.theory_uc = 
  Theory.add_decl my_theory decl_goal3  
\end{ocamlcode}

Addition of goal 4 is nothing more complex:
\begin{ocamlcode}
let decl_goal4 : Decl.decl = 
  Decl.create_prop_decl Decl.Pgoal goal_id4 fmla4
let my_theory : 
  Theory.theory_uc = Theory.add_decl my_theory decl_goal4  
\end{ocamlcode}

Finally, we close our theory under construction as follows.
\begin{ocamlcode}
let my_theory : Theory.theory = Theory.close_theory my_theory  
\end{ocamlcode}

We can inspect what we did by printing that theory:
\begin{ocamlcode}
let () = printf "@[theory is:@\n%a@]@." Pretty.print_theory my_theory
\end{ocamlcode}
which outputs
\begin{verbatim}
theory is:
theory My_theory
  (* use BuiltIn *)
  
  goal goal1 : true \/ false
  
  predicate A
  
  predicate B
  
  goal goal2 : A /\ B -> A
  
  (* use int.Int *)
  
  goal goal3 : (2 + 2) = 4
  
  goal goal4 : forall x:int. (x * x) >= 0
end
\end{verbatim}

From a theory, one can compute at once all the proof tasks it contains
as follows:
\begin{ocamlcode}
let my_tasks : Task.task list = 
  List.rev (Task.split_theory my_theory None None)
\end{ocamlcode}
Note that the tasks are returned in reverse order, so we reverse the
list above.

We can check our generated tasks by printing them:
\begin{ocamlcode}
let () = 
  printf "Tasks are:@.";
  let _ =
    List.fold_left
      (fun i t -> printf "Task %d: %a@." i Pretty.print_task t; i+1)
      1 my_tasks
  in ()
\end{ocamlcode}

One can run provers on those tasks exactly as we did above.

\section{Applying Transformations}

[TO BE COMPLETED]

\section{Writing New Functions on Terms}

[TO BE COMPLETED]
% pattern-matching on terms, opening a quantifier

\section{Proof Sessions}

See the example \verb|examples/use_api/create_session.ml| of the
distribution for an illustration on how to manipulate proof sessions
from an OCaml program.

\section{ML Programs}

There are two ways for building \whyml programs from OCaml. The first
is to build untyped syntax trees for such \whyml programs, and then
call the \why typing procedure to build typed declarations. The second
way is to directly build typed programs using smart constructors that
check well-typedness at each step.

The first approach, building untyped trees and then typing them, is
examplified in file \verb|examples/use_api/mlw_tree.ml| of the
distribution. The second approach is
examplified in file \verb|examples/use_api/mlw.ml|. The first approach
is significantly simpler to do since the internal typing mechanism
using regions remains implicit, whereas when one uses the second
approach one should care about such typing. On the other hand, the
second approach is more ``efficient'' in the sense that no
intermediate form needs to be built in memory.




%%% Local Variables:
%%% mode: latex
%%% TeX-PDF-mode: t
%%% TeX-master: "manual"
%%% End:


\part{Reference Manual}


\chapter{Compilation, Installation}
\label{sec:install}


In short, installation proceeds as follows.
\begin{flushleft}\ttfamily
  ./configure\\
  make\\
  make install \mbox{\rmfamily (as super-user)}
\end{flushleft}

\section{Installation Instructions from Source Distribution}

After unpacking the distribution, go to the newly created directory
\texttt{why3-\whyversion}. Compilation must start with a
configuration phase which is run as 
\begin{verbatim}
./configure
\end{verbatim}
This analyzes your current configuration and checks if requirements hold.
Compilation requires:
\begin{itemize}
\item The Objective Caml compiler, version 3.11.2 or higher. It is
  available as a binary package for most Unix distributions. For
  Debian-based Linux distributions, you can install the packages
\begin{verbatim}
ocaml ocaml-native-compilers
\end{verbatim}
It is also installable from sources, downloadable from the site
\url{http://caml.inria.fr/ocaml/}
\end{itemize}

\noindent
For some of the \why tools, additional OCaml libraries are needed:
\begin{itemize}
\item For the graphical interface, the Lablgtk2 library is needed.
  It provides OCaml
  bindings of the gtk2 graphical library. For Debian-based Linux
  distributions, you can install the packages
\begin{verbatim}
liblablgtk2-ocaml-dev liblablgtksourceview2-ocaml-dev
\end{verbatim}
It is also installable from sources, available from the site
\url{http://wwwfun.kurims.kyoto-u.ac.jp/soft/olabl/lablgtk.html}

\item For \texttt{why3 bench}, the OCaml bindings of the sqlite3 library
are needed.
For Debian-based Linux distributions, you can install the package
\begin{verbatim}
libsqlite3-ocaml-dev
\end{verbatim}
It is also installable from sources, available from the site
\url{http://ocaml.info/home/ocaml_sources.html#ocaml-sqlite3}
\end{itemize}


If you want to use the specific Coq features, \ie the Coq tactic
(Section~\ref{sec:coqtactic}) and Coq realizations
(Section~\ref{sec:realizations}), then Coq has to be installed before
\why. Look at the summary printed at the end of the configuration
script to check if Coq has been detected properly. Similarly, for
using PVS (Section~\ref{sec:pvs}) or Isabelle (Section~\ref{sec:isabelle}) to discharge proofs, PVS and Isabelle must be
installed before \why. You should check that those proof assistants
are correctly detected by the configure script.

When configuration is finished, you can compile \why.
\begin{verbatim}
make
\end{verbatim}
Installation is performed (as super-user if needed) using
\begin{verbatim}
make install
\end{verbatim}
Installation can be tested as follows:
\begin{enumerate}
\item install some external provers (see~Section~\ref{provers} below)
\item run \verb|why3 config --detect|
\item run some examples from the distribution, \eg you should
obtain the following:
\begin{verbatim}
$ cd examples
$ why3 replay logic/scottish-private-club
Opening session... done
Progress: 4/4
 1/1
Everything OK.
$ why3 replay programs/same_fringe
Opening session... done
Progress: 12/12
 3/3
Everything OK.
\end{verbatim}
\end{enumerate}

\section{Local Use, Without Installation}

It is not mandatory to install \why into system directories.
\why can be configured and compiled for local use as follows:
\begin{verbatim}
./configure --enable-local
make
\end{verbatim}
The \why executables are then available in the subdirectory
\texttt{bin/}. This directory can be added in your \texttt{PATH}.

\section{Installation of the \why API}
\label{sec:installlib}\index{API}

By default, the \why API is not installed. It can be installed using
\begin{flushleft}\ttfamily
make byte opt \\
make install-lib \mbox{\rmfamily (as super-user)}
\end{flushleft}

\section{Installation of External Provers}
\label{provers}

\why can use a wide range of external theorem provers. These need to
be installed separately, and then \why needs to be configured to use
them. There is no need to install automatic provers, \eg SMT solvers,
before compiling and installing \why.

For installation of external provers, please refer to the specific
section about provers on the Web page \url{http://why3.lri.fr/}.

For configuring \why to use the provers, follow instructions given in
Section~\ref{sec:why3config}.

\section{Multiple Versions of the Same Prover}

\why is able to use several versions of the same
prover, \eg it can use both CVC3 2.2 and CVC3 2.4.1 at the same time.
The automatic detection of provers looks for typical names for their
executable command, \eg \texttt{cvc3} for CVC3. However, if you
install several version of the same prover it is likely that you would
use specialized executable names, such as \texttt{cvc3-2.2} or
\texttt{cvc3-2.4.1}. To allow the \why detection process to recognize
these, you can use the option \verb|--add-prover| with the
\texttt{config} command, \eg
\index{add-prover@\verb+--add-prover+}
\begin{verbatim}
why3 config --detect --add-prover cvc3-2.4 /usr/local/bin/cvc3-2.4.1
\end{verbatim}
the first argument (here \verb|cvc3-2.4|) must be one of the class of
provers known in the file \verb|provers-detection-data.conf| typically
located in \verb|/usr/local/share/why3| after installation. See
Appendix~\ref{sec:proverdetecttiondata} for details.


\section{Session Update after Prover Upgrade}
\label{sec:uninstalledprovers}

If you happen to upgrade a prover, \eg installing CVC3 2.4.1 in place
of CVC3 2.2, then the proof sessions formerly recorded will still
refer to the old version of the prover. If you open one such a session
with the GUI, and replay the proofs, you will be asked to choose
between 3 options:
\begin{itemize}
\item Keep the former proofs as they are. They will be marked as
  ``archived''.
\item Upgrade the former proofs to an installed prover (typically a
  upgraded version). The corresponding proof attempts will become
  attached to this new prover, and marked as obsolete,
  to make their replay mandatory.
\item Copy the former proofs to an installed prover. This is a
  combination of the actions above: each proof attempt is duplicated,
  one with the former prover marked as archived, and one for the new
  prover marked as obsolete.
\end{itemize}

Notice that if the prover under consideration is an interactive one, then
the copy option will duplicate also the edited proof scripts, whereas
the upgrade-without-archive option will just reuse the former proof scripts.

Your choice between the three options above will be recorded, one for
each prover, in the \why configuration file. Within the GUI, you can
discard these choices via the \textsf{Preferences} dialog.

Outside the GUI, the prover upgrades are handled as follows. The
\texttt{replay} command will just ignore proof attempts marked as
archived\index{archived}.
Conversely, a non-archived proof attempt with a non-existent
prover will be reported as a replay failure. The
\texttt{session} command performs move or copy operations on
proof attempts in a fine-grained way, using filters, as detailed in
Section~\ref{sec:why3session}.


% pour l'instant on ne documente pas
% {que devient l'option -to-known-prover de why3session ?
%   (d'ailleurs documenté en tant que --convert-unknown ??) Pourrait-on
%   permettre à why3session d'appliquer les choix d'association
%   vieux-prover/nouveau-prouveur stockés par l'IDE ?}


%%% Local Variables:
%%% mode: latex
%%% TeX-PDF-mode: t
%%% TeX-master: "manual"
%%% End:


\chapter{Reference Manuals for the \why Tools}
\label{chap:manpages}

This chapter details the usage of each of the command-line tools
provided by the \why environment. The main command is \texttt{why3};
it acts as an entry-point to all the features of \why. It is invoked
as such
\begin{verbatim}
why3 [general options...] <command> [specific options...]
\end{verbatim}

The following commands are available:
\begin{description}
\item[\texttt{bench}] produces benchmarks.
\item[\texttt{config}] manages the user's configuration,
  including the detection of installed provers.
\item[\texttt{doc}] produces HTML versions of \why source codes.
\item[\texttt{execute}] performs a symbolic execution of \whyml
  input files.
\item[\texttt{extract}] generates an OCaml program corresponding to
  \whyml input files.
\item[\texttt{ide}] provides a graphical interface to display goals
  and to run provers and transformations on them.
\item[\texttt{prove}] reads \why and \whyml input files and calls
  provers, on the command-line.
\item[\texttt{realize}] generates interactive proof skeletons for
  \why input files.
\item[\texttt{replay}] replays the proofs stored in a session,
  for regression test purposes.
\item[\texttt{session}] dumps various informations from a proof
  session, and possibly modifies the session.
\item[\texttt{wc}] gives some token statistics about \why and \whyml
  source codes.
\end{description}

All these commands are also available as standalone executable files,
if needed.

The commands accept a common subset of command-line options. In
particular, option \verb|--help| displays the usage and options.
\begin{description}
\item[\texttt{-L \textsl{<dir>}}]
  adds \texttt{\textsl{<dir>}} in the load path, to search for theories.
  \index{L@\verb+-L+|see{\texttt{-{}-library}}}
\item[\texttt{-{}-library \textsl{<dir>}}]
  is the same as \verb|-L|.
  \index{library@\verb+--library+}
\item[\texttt{-C \textsl{<file>}}]
  reads the configuration from the given file.
  \index{C@\verb+-C+|see{\texttt{-{}-config}}}
\item[\texttt{-{}-config \textsl{<file>}}]
  is the same as \verb|-C|.
  \index{config@\verb+--config+}
\item[\texttt{-{}-extra-config \textsl{<file>}}]
  reads additional configuration from the given file.
  \index{extra-config@\verb+--extra-config+}
\item[\texttt{-{}-list-debug-flags}]
  lists known debug flags.
  \index{list-debug-flags@\verb+--list-debug-flags+}
\item[\texttt{-{}-debug-all}]
  sets all debug flags (except flags which change the behavior).
  \index{debug-all@\verb+--debug-all+}
\item[\texttt{-{}-debug \textsl{<flag>}}]
  sets a specific debug flag.
  \index{debug@\verb+--debug+}
\item[\texttt{-{}-help}]
  displays the usage and the exact list of options for the given tool.
  \index{help@\verb+--help+}
\end{description}

\section{The \texttt{config} Command}
\label{sec:why3config}

\why must be configured to access external provers. Typically, this is done
by running the \texttt{config} command.
This must be done each time a new prover is installed.%
\index{config@\texttt{config}}%
\index{configuration file}

The provers that \why attempts to detect are described in
the readable configuration file \texttt{provers-detection-data.conf}
of the \why data directory (\eg
\texttt{/usr/local/share/why3}). Advanced users may try to modify this
file to add support for detection of other provers. (In that case,
please consider submitting a new prover configuration on the bug
tracking system.)

The result of provers detection is stored in the user's
configuration file (\verb+~/.why3.conf+ or, in the case of local
installation, \verb+why3.conf+ in \why sources top directory). This file
is also human-readable, and advanced users may modify it in order to
experiment with different ways of calling provers, \eg different
versions of the same prover, or with different options.

The \texttt{config} command also detects the plugins installed in the \why
plugins directory (\eg \texttt{/usr/local/lib/why3/plugins}). A
plugin must register itself as a parser, a transformation or a
printer, as explained in the corresponding section.
\index{plugin}

If the user's configuration file is already present,
\texttt{config} will only reset unset variables to default value,
but will not try to detect provers.
The option \verb|--detect-provers| should be used to force
\why to detect again the available
provers and to replace them in the configuration file. The option
\verb|--detect-plugins| will do the same for plugins.
\index{detect-provers@\verb+--detect-provers+}
\index{detect-plugins@\verb+--detect-plugins+}

If a supported prover is installed under a name
that is not automatically recognized by \texttt{why3config},
the option \verb|--add-prover| will add a specified binary
to the configuration. For example, an Alt-Ergo executable
\verb|/home/me/bin/alt-ergo-trunk| can be added as follows:
\begin{verbatim}
why3 config --add-prover alt-ergo /home/me/bin/alt-ergo-trunk
\end{verbatim}
As the first argument, one should put a prover
identification string. The list of known prover identifiers
can be obtained by the option \verb|--list-prover-ids|.
\index{add-prover@\verb+--add-prover+}
\index{list-prover-ids@\verb+--list-prover-ids+}

\section{The \texttt{prove} Command}
\label{sec:why3ref}

\why is primarily used to call provers on goals contained in an
input file. By default, such a file must be written either in \why language
(extension \texttt{.why}) or in \whyml language (extension \texttt{.mlw}).
However, a dynamically loaded
plugin can register a parser for some other format of logical problems,
\eg TPTP or SMT-LIB.
\index{prove@\texttt{prove}}

The \texttt{prove} command executes the following steps:
\begin{enumerate}
\item Parse the command line and report errors if needed.
\item Read the configuration file using the priority defined in
  Section~\ref{sec:whyconffile}.
\item Load the plugins mentioned in the configuration. It will not
  stop if some plugin fails to load.
\item Parse and typecheck the given files using the correct parser in order
  to obtain a set of \why theories for each file. It uses
  the filename extension or the \verb|--format| option to choose
  among the available parsers. \verb|why3 --list-formats| lists
  the registered parsers.
  \index{list-formats@\verb+--list-formats+}
  \whyml modules are turned into
  theories containing verification conditions as goals.
\item Extract the selected goals inside each of the selected theories
  into tasks. The goals and theories are selected using options
  \verb|-G/--goal| and \verb|-T/--theory|. Option
  \verb|-T/--theory| applies to the previous file appearing on the
  command line. Option \verb|-G/--goal| applies to the previous theory
  appearing on the command line. If no theories are selected in a file,
  then every theory is considered as selected. If no goals are selected
  in a theory, then every goal is considered as selected.
  \index{G@\verb+-G+|see{\texttt{-{}-goal}}}
  \index{goal@\verb+--goal+}
  \index{T@\verb+-T+|see{\texttt{-{}-theory}}}
  \index{theory@\verb+--theory+}
\item Apply the transformations requested
  with \verb|-a/--apply-transform| in their order of appearance on the
  command line. \verb|why3 --list-transforms| lists the known
  transformations; plugins can add more of them.
  \index{a@\verb+-a+|see{\texttt{-{}-apply-transform}}}
  \index{apply-transform@\verb+--apply-transform+}
  \index{list-transforms@\verb+--list-transforms+}
\item Apply the driver selected with the \verb|-D/--driver| option,
  or the driver of the prover selected with the \verb|-P/--prover|
  option. \verb|why3 --list-provers| lists the known provers, \ie the ones
  that appear in the configuration file.
  \index{D@\verb+-D+|see{\texttt{-{}-driver}}}
  \index{driver@\verb+--driver+}
  \index{P@\verb+-P+|see{\texttt{-{}-prover}}}
  \index{prover@\verb+--prover+}
  \index{list-provers@\verb+--list-provers+}
\item If option \verb|-P/--prover| is given, call the selected prover
  on each generated task and print the results. If option
  \verb|-D/--driver| is given, print each generated task using
  the format specified in the selected driver.
\end{enumerate}

%\texttt{why3} calls the provers sequentially, use \texttt{why3bench} if *)
%you want to call the provers concurrently.  *)

\noindent
The provers can give the following output:
\begin{description}
\item[Valid] The goal is proved in the given context.
\item[Unknown] The prover has stopped its search.
\item[Timeout] The prover has reached the time limit.
\item[Failure] An error has occurred.
\item[Invalid] The prover knows the goal cannot be proved.
\end{description}
% \why can also be *)
% used to provide other informations : *)
% \begin{itemize} *)
% \item \texttt{print-namespace} print the namespace of the selected *)
%   theories *)
% \item TO BE COMPLETED *)
% \end{itemize} *)

%Option \verb|--bisect| changes the behavior of why3. With this
%option, \verb|-P/--prover| and \verb|-o/--output| must be given
%and a valid goal must be selected. The last step executed by why3 is
%replaced by computing a minimal set (in the great majority of the
%case) of declarations that still prove the goal. Currently it does not
%use any information provided by the prover; it calls the prover
%multiple times with reduced context. The minimal set of declarations is
%then written in the prover syntax into a file located in the directory
%given to the \verb|-o/--output| option.

\section{The \texttt{ide} Command}
\label{sec:ideref}

The basic usage of the GUI is described by the tutorial of
Section~\ref{sec:gui}. There are no specific command-line options,
apart from the common options detailed in introduction to this
chapter. However at least one anonymous argument must be specified on
the command line. More precisely, the first anonymous argument must be
the directory of the session. If the directory does not exist, it is
created. The other arguments should be existing files that are going
to be added to the session. For convenience,
if there is only one anonymous argument, it can be an existing file and
in this case the session directory is obtained by removing the extension
from the file name.

We describe the actions of the various menus and buttons of the
interface.
\index{ide@\texttt{ide}}

\subsection{Session}
\label{sec:idref:session}
\why stores in a session the way you achieve to prove goals that come
from a file (\texttt{.why}), from weakest-precondition (\texttt{.mlw}) or by other
means. A session stores which file you prove, by applying which
transformations, by using which prover. A proof attempt records the
complete name of a prover (name, version, optional attribute), the
time limit and memory limit given, and the result of the prover. The
result of the prover is the same as when you run the \texttt{prove} command. It
contains the time taken and the state of the proof:

\begin{description}
\item[Valid] The task is valid according to the prover. The
  goal is considered proved.
\item[Invalid] The task is invalid.
\item[Timeout] the prover exceeded the time limit.
\item[OufOfMemory] The prover exceeded the memory limit.
\item[Unknown] The prover cannot determine if the task
  is valid. Some additional information can be provided.
\item[Failure] The prover reported a failure.
\item[HighFailure] An error occurred while trying to call the
  prover, or the prover answer was not understood.
\end{description}

Additionally, a proof attempt can have the following attributes:

\begin{description}
\item[obsolete]\index{obsolete!proof attempt} The prover associated to
  that proof attempt has not been run on the current task, but on an
  earlier version of that task. You need to replay the proof
  attempt, \ie run the prover with the current task of the proof
  attempt, in order to update the answer of the prover and remove this
  attribute.
\item[archived]\index{archived!proof attempt} The proof attempt is not useful
  anymore; it is kept for history; no \why tools will select it by
  default. Section \ref{sec:uninstalledprovers} shows an example
  of this usage.
\end{description}

Generally, proof attempts are marked obsolete just after
the start of the user interface. Indeed, when you load a session in order to
modify it (not with \texttt{why3session info} for instance), \why
rebuilds the goals to prove by using the information provided in the
session. If you modify the original file (\texttt{.why}, \texttt{.mlw}) or if the
transformations have changed (new version of \why), \why will detect
that. Since the provers might answer differently on these new
proof obligations, the corresponding proof attempts are marked obsolete.

% non
% We say that a session is obsolete if new
% goals are made obsolete by this method during start-up.

% Claude: Alors la je ne vois pas pourquoi
% A session can
% be not obsolete even if it contains obsolete goals.

\subsection{Left toolbar actions}

\begin{description}
\item[Context] presents the context in which the other tools below will
  apply. If ``only unproved goals'' is selected, no action will ever
  be applied to an already proved goal.  If ``all goals'', then
  actions are performed even if the goal is already proved. The second
  choice allows to compare provers on the same goal.

\item[Provers] provide a button for each detected prover. Clicking on such a
  button starts the corresponding prover on the selected goal(s).

\item[Split] splits the current goal into subgoals if it is a
  conjunction of two or more goals. It corresponds to the
  \verb|split_goal_wp| transformation.

\item[Inline] replaces the head predicate symbol of the goal with its
  definition. It corresponds to the
  \verb|inline_goal| transformation.

\item[Edit] starts an editor on the selected task.

  For automatic provers, this allows to see the file sent to the
  prover.

  For interactive provers, this also allows to add or modify the
  corresponding proof script. The modifications are saved, and can be
  retrieved later even if the goal was modified.

\item[Replay] replays all the obsolete proofs.

  If ``only unproved goals'' is selected, only formerly successful
  proofs are rerun. If ``all goals'', then all obsolete proofs are
  rerun, successful or not.

\item[Remove] removes a proof attempt or a transformation.

\item[Clean] removes any unsuccessful proof attempt for which there is
  another successful proof attempt for the same goal

\item[Interrupt] cancels all the proof attempts currently scheduled
  but not yet started.

\end{description}

\subsection{Menus}

\begin{description}
\item[Menu \textsf{File}]\emptyitem
\begin{description}
\item[Add File] adds a file in the GUI.
%\item[Detect provers] runs provers auto-detection
\item[Preferences] opens a window for modifying preferred
  configuration parameters, see details below.
\item[Reload] reloads the input files from disk, and update the session state accordingly.
\item[Save session] saves current session state on disk. The policy to decide when to save the session is configurable, as described in the preferences below.
\item[Quit] exits the GUI.
\end{description}

\item[Menu \textsf{View}]\emptyitem
\begin{description}
\item[Expand All] expands all the rows of the tree view.
\item[Collapse proved goals] closes all the rows of the tree view
  that are proved.
% \item[Hide proved goals] completely hides the proved rows of the tree
%   view [EXPERIMENTAL]
\end{description}

\item[Menu \textsf{Tools}]
A copy of the tools already available in the left toolbar, plus:
\begin{description}
\item[Mark as obsolete] marks all the proof as
  obsolete.
  This allows to replay every proof.
\item[Non-splitting transformation] applies one of the available
  transformations, as listed in Section~\ref{sec:transformations}.
\item[Splitting transformation] is the same as above, but for
  splitting transformations, \ie those that can generate
  several sub-goals.
\end{description}

\item[Menu \textsf{Help}]
A very short online help, and some information about this software.
\end{description}

\subsection{Preferences Dialog}

The preferences dialog allows you to customize various settings. They
are grouped together under several tabs.

\begin{description}
\item[\textsf{General Settings} tab] allows one to set
  various general settings.
\begin{itemize}
\item the limits set on resource usages:
  \begin{itemize}
  \item the time limit given to provers, in seconds
  \item the memory given to provers, in megabytes
  \item the maximal number of simultaneous provers allowed to run in parallel
  \end{itemize}
  By default, modification of any of these settings has effect only
  for the current run of the GUI. A checkbox allows you to save these
  settings also for future sessions.
\item a few display settings:
  \begin{itemize}
  \item introduce premises: if selected, the goal of the task shown in
    top-right window is displayed after introduction of universally
    quantified variables and implications, \eg a goal of the form
    $\forall x: t. P \rightarrow Q$ is displayed as
    \[
    \begin{array}{l}
      x : t \\
      H : P \\
      \hline
      Q
    \end{array}
    \]
  \item show labels in formulas
  \item show source locations in formulas
  \item show time limit for each proof
  \end{itemize}
\item the policy for saving session:
  \begin{itemize}
  \item always save on exit (default): the current state of the proof session is saving on exit
  \item never save on exit: the current state of the session is never saved
    automatically, you must use menu \textsf{File/Save session}
  \item ask whether to save: on exit, a popup window asks whether you
    want to save or not.
  \end{itemize}
\end{itemize}
\item[\textsf{Editors} tab] allows one to customize the use
  of external editors for proof scripts.
\begin{itemize}
\item The default editor to use when the \textsf{Edit} button is
  pressed.
  \urldef{\urlprfgen}{\url}{http://proofgeneral.inf.ed.ac.uk/}
\item For each installed prover, a specific editor can be selected to
  override the default. Typically if you install the Coq prover, then
  the editor to use will be set to ``CoqIDE'' by default, and this
  dialog allows you to select the Emacs editor and its
 \ahref{\urlprfgen}{Proof General} mode instead%
 \begin{latexonly} (\urlprfgen)\end{latexonly}.
\end{itemize}
\item[\textsf{Provers} tab]
  allows to select which of the installed provers one wants to see
  as buttons in the left toolbar.
\item[\textsf{Uninstalled Provers} tab] presents all the
  decision previously taken for missing provers, as described in
  Section~\ref{sec:uninstalledprovers}. You can remove any recorded
  decision by clicking on it.
\end{description}


\section{The \texttt{bench} Command}

The \texttt{bench} command adds a scheduler on top of the \why
library. It is designed to compare various components
of automatic proofs: automatic provers, transformations, definitions
of a theory. For that purpose, it tries to prove predefined goals using
each component to compare. Various formats can be used as outputs:
\begin{description}
\item[\texttt{csv}] the simpler and more informative format; the results are
  represented in an array; the rows correspond to the
  compared components, the columns correspond to the result
  (Valid, Unknown, Timeout, Failure, Invalid) and the CPU time taken in seconds.
\item[\texttt{average}] it summarizes the number of the five different answers
  for each component. It also gives the average time taken.
\item[\texttt{timeline}] for each component, it gives the number of valid goals
  along the time (10 slices between 0 and the longest time a component
  takes to prove a goal)
\end{description}

The compared components can be defined in an \emph{rc-file};
\texttt{examples/misc/prgbench.rc} is an example of such a file. More
generally, a bench configuration file looks like
\begin{verbatim}
[probs "myprobs"]
   file = "examples/mygoal.why" #relatives to the rc file
   file = "examples/myprogram.mlw"
   theory = "myprogram.T"
   goal = "mygoal.T.G"

   transform = "split_goal" #applied in this order
   transform = "..."
   transform = "..."

[tools "mytools"]
   prover = cvc3
   prover = altergo
   #or only one
   driver = "..."
   command = "..."

   loadpath = "..." #added to the one in why3.conf
   loadpath = "..."

   timelimit = 30
   memlimit = 300

   use = "toto.T" #use the theory toto (allow to add metas)

   transform = "simplify_array" #only 1 to 1 transformation

[bench "mybench"]
   tools = "mytools"
   tools = ...
   probs = "myprobs"
   probs = ...
   timeline = "prgbench.time"
   average = "prgbench.avg"
   csv = "prgbench.csv"
\end{verbatim}

Such a file can define three families \texttt{tools}, \texttt{probs},
\texttt{bench}. A \texttt{tools} section defines a set of components to
compare. A \texttt{probs} section defines a set of goals on which to compare some
components. A \texttt{bench} section defines which components to
compare using which goals. It refers by name to some sections
\texttt{tools} and \texttt{probs} defined in the same file. The order
of the definitions is irrelevant. Notice that one can use
\texttt{loadpath} in a \texttt{tools} section to compare different
axiomatizations.

One can run all the bench given in one bench configuration file with
\begin{verbatim}
why3 bench -B path_to_my_bench.rc
\end{verbatim}

\section{The \texttt{replay} Command}
\label{sec:why3replayer}

The \texttt{replay} command is meant to execute the proofs
stored in a \why session file, as produced by the IDE. Its
main purpose is to play non-regression tests. For instance,
\texttt{examples/regtests.sh} is a script that runs regression tests on
all the examples.
\index{replay@\texttt{replay}}

The tool is invoked in a terminal or a script using
\begin{flushleft}\ttfamily
  why3 replay \textsl{[options] <project directory>}
\end{flushleft}
The session file \texttt{why3session.xml} stored in the given
directory is loaded and all the proofs it contains are rerun. Then,
all the differences between the information stored in the session file and
the new run are shown.

Nothing is shown when there is no change in the results, whether the
considered goal is proved or not. When all the proof
are done, a summary of what is proved or not is displayed using a
tree-shape pretty print, similar to the IDE tree view after doing
``Collapse proved goals''. In other words, when a goal, a theory, or a
file is fully proved, the subtree is not shown.

\paragraph{Obsolete proofs}

When some proof attempts stored in the session file are
obsolete\index{obsolete!proof attempt},
the replay is run anyway, as with the replay button in the IDE. Then, the session
file will be updated if both
\begin{itemize}
\item all the replayed proof attempts give the same result as what
  is stored in the session
\item every goals are proved.
\end{itemize}
In other cases, you can use the IDE to update the session, or use the
option \verb|--force| described below.

\paragraph{Exit code and options}

The exit code is 0 if no difference was detected, 1 if there
was. Other exit codes mean some failure in running the replay.

Options are:
\begin{description}
\item[\texttt{-s}] suppresses the output of the final tree view.
\item[\texttt{-q}] runs quietly (no progress info).
\item[\texttt{-{}-force}] enforces saving the session, if all proof
  attempts replayed correctly, even if some goals are not proved.
\item[\texttt{-{}-obsolete-only}] replays the proofs only if the session
  contains obsolete proof attempts.
\item[\texttt{-{}-smoke-detector \{none|top|deep\}}] tries to detect
  if the context is self-contradicting.
\item[\texttt{-{}-prover \textsl{<prover>}}] restricts the replay to the
  selected provers only.
\end{description}

\paragraph{Smoke detector}

The smoke detector tries to detect if the context is
self-contradicting and, thus, that anything can be proved in this
context. The smoke detector can't be run on an outdated session and does
not modify the session.  It has three possible configurations:
\begin{description}
\item[\texttt{none}] Do not run the smoke detector.
\item[\texttt{top}] The negation of each proved goal is sent with the
  same timeout to the prover that proved the original goal.
\begin{verbatim}
  Goal G : forall x:int. q x -> (p1 x \/ p2 x)
\end{verbatim}
  becomes
\begin{verbatim}
  Goal G : ~ (forall x:int. q x -> (p1 x \/ p2 x))
\end{verbatim}
  In other words, if the smoke detector is triggered, it means that the context
  of the goal \texttt{G} is self-contradicting.
\item[\texttt{deep}] This is the same technique as \texttt{top} but
  the negation is pushed under the universal quantification (without
  changing them) and under the implication. The previous example
  becomes
\begin{verbatim}
  Goal G : forall x:int. q x /\ ~ (p1 x \/ p2 x)
\end{verbatim}
  In other words, the premises of goal \texttt{G} are pushed in the
  context, so that if the smoke detector is triggered, it means that
  the context of the goal \texttt{G} and its premises are
  self-contradicting. It should be clear that detecting smoke in that
  case does not necessarily means that there is a mistake: for
  example, this could occur in the WP of a program with an unfeasible
  path.
\end{description}

At the end of the replay, the name of the goals that triggered the
smoke detector are printed:
\begin{verbatim}
  goal 'G', prover 'Alt-Ergo 0.93.1': Smoke detected!!!
\end{verbatim}
Moreover \texttt{Smoke detected} (exit code 1) is printed at the end
if the smoke detector has been triggered, or \texttt{No smoke
  detected} (exit code 0) otherwise.



\section{The \texttt{session} Command}
\label{sec:why3session}

The \texttt{session} command makes it possible to extract information from
proof sessions on the command line, or even modify them to some
extent. The invocation of this program is done under the form
\begin{verbatim}
why3 session <subcommand> [options] <session directories>
\end{verbatim}
The available subcommands are as follows:
\begin{description}
\item[\texttt{info}] prints informations and statistics about sessions.
\item[\texttt{latex}] outputs session contents in LaTeX format.
\item[\texttt{html}] outputs session contents in HTML format.
\item[\texttt{mod}] modifies some of the proofs, selected by a filter.
\item[\texttt{copy}] duplicates some of the proofs, selected by a filter.
\item[\texttt{copy-archive}] same as copy but also archives the
  original proofs\index{archived!proof attempt}.
\item[\texttt{rm}] removes some of the proofs, selected by a filter.
\end{description}

The first three commands do not modify the sessions, whereas the last
four modify them. Only the proof attempts recorded are modified. No
prover is called on the modified or created proof attempts, and
consequently the proof status is always marked as obsolete.

\subsection{Command \texttt{info}}

The command \texttt{why3 session info} reports various informations
about the session, depending on the following specific options.
\begin{description}
\item[\texttt{-{}-provers}] prints the provers that appear inside
  the session, one by line.
\item[\texttt{-{}-edited-files}] prints all the files that appear in
  the session as edited proofs.
\item[\texttt{-{}-stats}] prints various proofs statistics, as
  detailed below.
\item[\texttt{-{}-tree}] prints the structure of the session as a
  tree in ASCII, as detailed below.
\item[\texttt{-{}-print0}] separates the results of the options
  \verb|provers| and \verb|--edited-files| by the character number 0
  instead of end of line \verb|\n|. That allows you to safely use
  (even if the filename contains space or carriage return) the result
  with other commands. For example you can count the number of proof
  line in all the coq edited files in a session with:
\begin{verbatim}
why3 session info --edited-files vstte12_bfs --print0 | xargs -0 coqwc
\end{verbatim}
  or you can add all the edited files in your favorite repository
  with:
\begin{verbatim}
why3 session info --edited-files --print0 vstte12_bfs.mlw | \
    xargs -0 git add
\end{verbatim}

\end{description}

\paragraph{Session Tree}

The hierarchical structure of the session is printed as a tree in
ASCII. The files, theories, goals are marked with a question mark
\verb|?|, if they are not verified. A proof is usually said to be
verified if the proof result is \verb|valid| and the proof is not
obsolete.
However here specially we separate these two properties. On
the one hand if the proof suffers from an internal failure we mark it
with an exclamation mark \verb|!|, otherwise if it is not valid we
mark it with a question mark \verb|?|, finally if it is valid we add
nothing. On the other hand if the proof is obsolete we mark it with an
\verb|O| and if it is archived we mark it with an \verb|A|.

For example, here are the session tree produced on the ``hello
proof'' example of Section~\ref{chap:starting}.
{\scriptsize
\begin{verbatim}
hello_proof---../hello_proof.why?---HelloProof?-+-G3-+-Simplify (1.5.4)?
                                                |    `-Alt-Ergo (0.94)
                                                |-G2?-+-split_goal?-+-G2.2-+-Simplify (1.5.4)
                                                |     |             |      `-Alt-Ergo (0.94)
                                                |     |             `-G2.1?-+-Coq (8.3pl4)?
                                                |     |                     |-Simplify (1.5.4)?
                                                |     |                     `-Alt-Ergo (0.94)?
                                                |     |-Simplify (1.5.4)?
                                                |     `-Alt-Ergo (0.94)?
                                                `-G1---Simplify (1.5.4)
\end{verbatim}
}

\paragraph{Session Statistics}

The proof statistics given by option \verb|--stats| are as follows:
\begin{itemize}
\item Number of goals: give both the total number of goals, and the
  number of those that are proved (possibly after a transformation).
\item Goals not proved: list of goals of the session which are not
  proved by any prover, even after a transformation.
\item Goals proved by only one prover: the goals for which there is only
  one successful proof. For each of these, the prover which was
  successful is printed. This also includes the sub-goals generated by
  transformations.
\item Statistics per prover: for each of the prover used in the
  session, the number of proved goals is given. This also includes the
  sub-goals generated by transformations. The respective minimum,
  maximum and average time and on average running time is
  shown. Beware that these time data are computed on the
  goals \emph{where the prover was successful}.
\end{itemize}

For example, here are the session statistics produced on the ``hello
proof'' example of Section~\ref{chap:starting}.
{\footnotesize
\begin{verbatim}
== Number of goals ==
  total: 5  proved: 3

== Goals not proved ==
  +-- file ../hello_proof.why
    +-- theory HelloProof
      +-- goal G2
        +-- transformation split_goal
          +-- goal G2.1

== Goals proved by only one prover ==
  +-- file ../hello_proof.why
    +-- theory HelloProof
      +-- goal G1: Simplify (1.5.4) (0.00)
      +-- goal G3: Alt-Ergo (0.94) (0.00)

== Statistics per prover: number of proofs, time (minimum/maximum/average) in seconds ==
  Alt-Ergo (0.94)      :   2   0.00   0.00   0.00
  Simplify (1.5.4)     :   2   0.00   0.00   0.00

\end{verbatim}
}

\subsection{Command \texttt{latex}}

Command \texttt{latex} produces a summary of the replay under the form
of a tabular environment in LaTeX, one tabular for each theory, one
per file.

The specific options are
\begin{description}
\item[\texttt{-style \textsl{<n>}}] sets output style (1 or 2, default 1)
  Option \texttt{-style 2} produces an alternate version of LaTeX
  output, with a different layout of the tables.
\item[\texttt{-o \textsl{<dir>}}] indicates where
  to produce LaTeX files (default: the session directory).
\item[\texttt{-longtable}] uses the `longtable' environment instead of
  `tabular'.
\item[\texttt{-e \textsl{<elem>}}] produces a table for the given element, which is
  either a file, a theory or a root goal. The element must be specified
  using its path in dot notation, \eg \verb|file.theory.goal|. The
  file produced is named accordingly,
  \eg \verb|file.theory.goal.tex|.  This option can be given several
  times to produce several tables in one run. When this option is
  given at least once, the default behavior that is to produce one
  table per theory is disabled.
\end{description}

\paragraph{Customizing LaTeX output}

The generated LaTeX files contain some macros that must be defined
externally.  Various definitions can be given to them to customize the
output.
\begin{description}
\item[\texttt{\bs{}provername}] macro with one parameter, a prover name
\item[\texttt{\bs{}valid}] macro with one parameter, used where the corresponding prover answers that the goal is valid. The parameter is the time in seconds.
\item[\texttt{\bs{}noresult}] macro without parameter, used where no result
  exists for the corresponding prover
\item[\texttt{\bs{}timeout}] macro without parameter, used where the corresponding prover reached the time limit
\item[\texttt{\bs{}explanation}] macro with one parameter, the goal name or its explanation
\end{description}

\begin{figure}[t]
\begin{center}
\lstinputlisting[basicstyle={\ttfamily\small}]{./replayer_macros.tex}
\end{center}
\caption{Sample macros for the LaTeX command}
\label{fig:custom-latex}
\end{figure}

\begin{figure}[t]
\begin{center}
%HEVEA\begin{toimage}
\input{HelloProof.tex}
%HEVEA\end{toimage}
%HEVEA\imageflush
\end{center}
\caption{LaTeX table produced for the HelloProof example (style 1)}
\label{fig:latex}
\end{figure}

\begin{figure}[t]
\begin{center}
%HEVEA\begin{toimage}
\input{HelloProof-style2.tex}
%HEVEA\end{toimage}
%HEVEA\imageflush
\end{center}
\caption{LaTeX table produced for the HelloProof example (style 2)}
\label{fig:latexstyle2}
\end{figure}

Figure~\ref{fig:custom-latex} suggests some definitions for these
macros, while Figures~\ref{fig:latex} and~\ref{fig:latexstyle2} show
the tables obtained from the HelloProof example of
Section~\ref{chap:starting}, respectively with style 1 and 2.

\subsection{Command \texttt{html}}

This command produces a summary of the proof session in HTML syntax.
There are three styles of output: `table', `simpletree', and
`jstree'. The default is `table'.

The file generated is named \texttt{why3session.html} and is written
in the session directory by default (see option \texttt{-o} to
override this default).

\begin{figure}[t]
%BEGIN LATEX
\begin{center}
\fbox{\includegraphics[width=0.9\textwidth]{hello_proof.png}}
\end{center}
%END LATEX
\begin{htmlonly}
\begin{rawhtml}
<h1>Why3 Proof Results for Project "hello_proof"</h1>
<h2><font color="#FF0000">Theory "HelloProof": not fully verified</font></h2>
<table border="1"><tr><td colspan="2">Obligations</td><td text-rotation="90">Alt-Ergo (0.94)</td><td text-rotation="90">Coq (8.3pl4)</td><td text-rotation="90">Simplify (1.5.4)</td></td></tr>
<td bgcolor="#C0FFC0" colspan="2">G1</td><td bgcolor="#E0E0E0">---</td><td bgcolor="#E0E0E0">---</td><td bgcolor="#C0FFC0">0.00</td></tr>
<td bgcolor="#FF0000" colspan="2">G2</td><td bgcolor="#FF8000">0.00</td><td bgcolor="#E0E0E0">---</td><td bgcolor="#FF8000">0.00</td></tr>
<tr><td bgcolor="#FF0000" colspan="2">split_goal</td><td bgcolor="#E0E0E0"></td><td bgcolor="#E0E0E0"></td><td bgcolor="#E0E0E0"></td></tr>
<td rowspan="2">&nbsp;&nbsp;</td><td bgcolor="#FF0000" colspan="1">1.</td><td bgcolor="#FF8000">0.00</td><td bgcolor="#FF8000">0.43</td><td bgcolor="#FF8000">0.00</td></tr>
<tr><td bgcolor="#C0FFC0" colspan="1">2.</td><td bgcolor="#C0FFC0">0.00</td><td bgcolor="#E0E0E0">---</td><td bgcolor="#C0FFC0">0.00</td></tr>
<td bgcolor="#C0FFC0" colspan="2">G3</td><td bgcolor="#C0FFC0">0.00</td><td bgcolor="#E0E0E0">---</td><td bgcolor="#FF8000">0.00</td></tr>
</table>
\end{rawhtml}
\end{htmlonly}
\caption{HTML table produced for the HelloProof example}
\label{fig:html}
\end{figure}

The style `table' outputs the contents of the session as a table,
similar to the LaTeX output above. Figure~\ref{fig:html} is the HTML
table produced for the `HelloProof' example, as typically shown in a
Web browser. The gray cells filled with \texttt{---} just mean that
the prover was not run on the corresponding goal. Green background
means the result was ``Valid'', other cases are in orange
background. The red background for a goal means that the goal was not
proved.

The style `simpletree' displays the contents of the session under the
form of tree, similar to the tree view in the IDE. It uses only basic
HTML tags such as \verb|<ul>| and \verb|<li>|.

The style `jstree' displays a dynamic tree view of the session, where
you can click on various parts to expand or shrink some parts of the
tree. Clicking on an edited proof script also shows the contents of
this script. Technically, it uses the `jstree' plugin of the javascript
library `jquery'.

Specific options for this command are as follows.
\begin{description}
\item[\texttt{-{}-style \textsl{<style>}}] sets the style to use, among
  \texttt{simpletree}, \texttt{jstree}, and \texttt{table}; defaults to
  \texttt{table}.

\item[\texttt{-o \textsl{<dir>}}] sets the directory where to output
  the produced files (`\texttt{-}' for stdout). The default is to output
  in the same directory as the session itself.

\item[\texttt{-{}-context}] adds context around the generated code in
  order to allow direct visualization (header, css, ...). It also adds
  in the output directory all the needed external files. It can't be set with
  stdout output.

\item[\texttt{-{}-add\_pp \textsl{<suffix>} \textsl{<cmd>} \textsl{<out\_suffix>}}] sets a specific
  pretty-printer for files with the given suffix. Produced files use
  \texttt{\textsl{<out\_suffix>}} as suffix. \texttt{\textsl{<cmd>}} must contain
  `\texttt{\%i}' which will be replaced by the input file and
  `\texttt{\%o}' which will be replaced by the output file.

\item[\texttt{-{}-coqdoc}] uses the \verb|coqdoc| command to display Coq proof
  scripts. This is equivalent to \texttt{-{}-add\_pp .v "coqdoc
    -{}-no-index -{}-html -o \%o \%i" .html}

\end{description}

\subsection{Commands modifying the proof attempts}

The commands \texttt{mod}, \texttt{copy}, \texttt{copy-archive},
and \texttt{rm}, share the same set of options for selecting the proof
attempts to work on:
\begin{description}
\item[\texttt{-{}-filter-prover \textsl{<prover>}}] selects only the proof attempt with
  the given prover. This option can be specified multiple times in order
  to select all the proofs that corresponds to any of the given
  provers.
\item[\texttt{-{}-filter-verified yes}] selects only
  the proofs that are valid and not obsolete, while option
  \verb|--filter-verified no| selects the ones that are not verified.
  \verb|--filter-verified all|, the default, does not perform such a selection.
\item[\texttt{-{}-filter-verified-goal yes}] restricts the selection
  to the proofs of verified goals (that does not mean that the proof is
  verified). Same for the other cases \verb|no| and \verb|all|.
\item[\texttt{-{}-filter-archived yes}] restricts the selection
  to the proofs that are archived.
  Same for the other cases \verb|no|
  and \verb|all| except the default is \verb|no|.
\end{description}

\noindent
The commands \texttt{mod}, \texttt{copy}, and \texttt{copy-archive},
share the same set of options to specify the modification. The
command \texttt{mod} modifies directly the proof attempt,
\texttt{copy} copies the proof attempt before doing the modification,
\texttt{copy-archive} marks the original proof attempt as
archived.
The options are:
\begin{description}
\item[\texttt{-{}-set-obsolete}] marks the selected proofs as
  obsolete.
\item[\texttt{-{}-set-archived}] marks the selected proofs as archived.
\item[\texttt{-{}-unset-archived}] removes the archived attribute
  from the selected proofs.
\item[\texttt{-{}-to-prover \textsl{<prover>}}] modifies the prover, for example
  \texttt{-{}-to-prover Alt-Ergo,0.94}. A conflict arises if a proof
  with this prover already exists. In this case, you can choose between four
  behaviors:
\begin{itemize}
\item replace the proof (\verb|-f|, \verb|--force|);
\item do not replace the proof (\verb|-n|, \verb|--never|);
\item replace the proof unless it is verified (valid and not
  obsolete) (\verb|-c|, \verb|--conservative|); this is the default;
\item ask the user each time the case arises (\verb|-i|, \verb|--interactive|).
\end{itemize}
\end{description}

The command \texttt{rm} removes the selected proof
attempts. The options \verb|--interactive|, \verb|--force|, and
\verb|--conservative|, can also be used to respectively ask before
each suppression, suppress all the selected proofs (default), and remove
only the proofs that are not verified. The macro option \verb|--clean|
corresponds to \verb|--filter-verified-goal --conservative| and
removes the proof attempts that are not verified but which correspond
to verified goals.

The commands of this section do not accept by default to modify an
obsolete session (as defined in \ref{sec:idref:session}). You need to
add the option \verb|-F| to force this behavior.


% pour l'instant on ne documente pas parce que commenté dans le code
% \todo{A adapter en fonction de la decision sur l'upgrade de prover}

% If you just want to update one session with updated provers you can
% use \verb|--convert-unknown| instead of the option \verb|--to-prover|.
% \begin{verbatim}
% why3session copy  --convert-unknown
% \end{verbatim}
% For each proof attempt associated to an unknown prover (a prover not in
% \verb|.why3.conf|) and not archived, it will try to find a known prover
% with the same name. If it finds one, the proof attempt is copied to this
% prover and the old proof is set to archived. The corresponding edited
% files, if any, are copied and regenerated for the new prover An archived
% proof is not replayed by why3replayer.



\section{The \texttt{doc} Command}
\label{sec:why3doc}

This tool can produce HTML pages from \why source code.
\why code for theories or modules is output in
preformatted HTML code. Comments are interpreted in three different ways.
\begin{itemize}
\item Comments starting with at least three stars are completed
  ignored.
\item Comments starting with two stars are interpreted as textual
  documentation. Special constructs are interpreted as described
  below. When the previous line is not empty, the comment is indented to
  the right, so as to be displayed as a description of that line.
\item Comments starting with one star only are interpreted as code
  comments, and are typeset as the code
\end{itemize}

Additionally, all the \why identifiers are typeset with links so that
one can navigate through the HTML documentation, going from some
identifier use to its definition.

\paragraph{Options}

\begin{description}
\item[\texttt{-o \textsl{<dir>}}] defines the directory where to
  output the HTML files.
\item[\texttt{-{}-output \textsl{<dir>}}] is the same as \verb|-o|.
\item[\texttt{-{}-index}] generates an index file \texttt{index.html}.
  This is the default behavior if more than one file
  is passed on the command line.
\item[\texttt{-{}-no-index}] prevents the generation of an index file.
\item[\texttt{-{}-title \textsl{<title>}}] sets title of the
  index page.
\item[\texttt{-{}-stdlib-url \textsl{<url>}}] sets a URL for files
  found in load path, so that links to definitions can be added.
\end{description}

\paragraph{Typesetting textual comments}

Some constructs are interpreted:
\begin{itemize}
\item \texttt{\{\textsl{c text}\}} interprets character \textsl{c} as
  some typesetting command:
  \begin{description}
  \item[1-6] a heading of level 1 to 6 respectively
  \item[h] raw HTML
  \end{description}
\item \texttt{[\textsl{code}]} is a code escape: the text
  \textsl{code} is typeset as \why code.
\end{itemize}

A CSS file \verb|style.css| suitable for rendering is generated in the
same directory as output files. This CSS style can be modified manually,
since regenerating the HTML documentation will not overwrite an existing
\verb|style.css| file.

\section{The \texttt{execute} Command}
\label{sec:why3execute}

\why can symbolically execute programs written using the \whyml language
(extension \texttt{.mlw}). See also Section~\ref{sec:execute}.
\index{execute@\texttt{execute}}

\section{The \texttt{extract} Command}
\label{sec:why3extract}

\why can extract programs written using the \whyml language
(extension \texttt{.mlw}) to OCaml. See also Section~\ref{sec:extract}.
\index{extract@\texttt{extract}}

\section{The \texttt{realize} Command}
\label{sec:why3realize}

\why can produce skeleton files for proof assistants that, once filled,
realize the given theories. See also Section~\ref{sec:realizations}.
\index{realize@\texttt{realize}}

\section{The \texttt{wc} Command}
\label{sec:why3wc}

\why can give some token statistics about \why and \whyml source codes.
\index{wc@\texttt{wc}}

%%% Local Variables:
%%% mode: latex
%%% TeX-PDF-mode: t
%%% TeX-master: "manual"
%%% End:


\chapter{Language Reference}
\label{chap:syntaxref}

This chapter gives the grammar and semantics for \why and \whyml input files.

\section{Lexical Conventions}

Lexical conventions are common to \why and \whyml.

% TODO: blanks

\subsection{Comments}

Comments are enclosed by \texttt{(*} and \texttt{*)} and can be nested.

\subsection{Strings}

Strings are enclosed in double quotes (\verb!"!). Double quotes can be
escaped in strings using the backslash character (\verb!\!).
The other special sequences are \verb!\n! for line feed and \verb!\t!
for horizontal tab.
In the following, strings are referred to with the non-terminal
\nonterm{string}{}.

\subsection{Identifiers}

The syntax distinguishes lowercase and
uppercase identifiers and, similarly, lowercase and uppercase
qualified identifiers.

\begin{center}\framebox{\input{./qualid_bnf.tex}}\end{center}

\subsection{Constants}
The syntax for constants is given in Figure~\ref{fig:bnf:constant}.
Integer and real constants have arbitrary precision.
Integer constants may be given in base 16, 10, 8 or 2.
Real constants may be given in base 16 or 10.

\begin{figure}
\begin{center}\framebox{\input{./constant_bnf.tex}}\end{center}
  \caption{Syntax for constants.}
\label{fig:bnf:constant}
\end{figure}

\subsection{Operators}

Prefix and infix operators are built from characters organized in four
categories (\textsl{op-char-1} to \textsl{op-char-4}).
\begin{center}\framebox{\input{./operator_bnf.tex}}\end{center}
Infix operators are classified into 4 categories, according to the
characters they are built from:
\begin{itemize}
\item level 4: operators containing only characters from
\textit{op-char-4};
\item level 3: those containing
 characters from \textit{op-char-3} or \textit{op-char-4};
\item level 2: those containing
 characters from \textit{op-char-2}, \textit{op-char-3} or
 \textit{op-char-4};
\item level 1: all other operators (non-terminal \textit{infix-op-1}).
\end{itemize}

\subsection{Labels}

Identifiers, terms, formulas, program expressions
can all be labeled, either with a string, or with a location tag.
\begin{center}\framebox{\input{./label_bnf.tex}}\end{center}
A location tag consists of a file name, a line number, and starting
and ending characters.

%%%%%%%%%%%%%%%%%%%%%%%%%%%%%%%%%%%%%%%%%%%%%%%%%%%%%%%%%%%%%%%%%%%%%%%%%%%%%%

\section{The \why Language}

\subsection{Terms}

The syntax for terms is given in Figure~\ref{fig:bnf:term}.
The various constructs have the following priorities and
associativities, from lowest to greatest priority:
\begin{center}
  \begin{tabular}{|l|l|}
    \hline
    construct & associativity \\
    \hline\hline
    \texttt{if then else} / \texttt{let in} & -- \\
    label & -- \\
    cast  & -- \\
    infix-op level 1 & left \\
    infix-op level 2 & left \\
    infix-op level 3 & left \\
    infix-op level 4 & left \\
    prefix-op     & --   \\
    function application & left \\
    brackets / ternary brackets & -- \\
    bang-op       & --   \\
    \hline
  \end{tabular}
\end{center}

Note the curryfied syntax for function application, though partial
application is not allowed (rejected at typing).

\begin{figure}
  \begin{center}\framebox{\input{./term_bnf.tex}}\end{center}
  \caption{Syntax for terms.}
\label{fig:bnf:term}
\end{figure}

\subsection{Type Expressions}

The syntax for type expressions is the following:
\begin{center}\framebox{\input{./type_bnf.tex}}\end{center}
Built-in types are \texttt{int}, \texttt{real}, and tuple types.
Note that the syntax for type
expressions notably differs from the usual ML syntax (\eg the
type of polymorphic lists is written \texttt{list 'a}, not \texttt{'a list}).

\subsection{Formulas}

The syntax for formulas is given Figure~\ref{fig:bnf:formula}.
The various constructs have the following priorities and
associativities, from lowest to greatest priority:
\begin{center}
  \begin{tabular}{|l|l|}
    \hline
    construct & associativity \\
    \hline\hline
    \texttt{if then else} / \texttt{let in} & -- \\
    label & -- \\
    \texttt{->} / \texttt{<->} & right \\
    \verb!\/! / \verb!||! & right \\
    \verb|/\| / \verb!&&! & right \\
    \texttt{not}  & -- \\
    infix level 1 & left \\
    infix level 2 & left \\
    infix level 3 & left \\
    infix level 4 & left \\
    prefix        & --   \\
    \hline
  \end{tabular}
\end{center}
Note that infix symbols of level 1 include equality (\texttt{=}) and
disequality (\texttt{<>}).

\begin{figure}
  \begin{center}\framebox{\input{./formula_bnf.tex}}\end{center}
  \caption{Syntax for formulas.}
\label{fig:bnf:formula}
\end{figure}

Notice that there are two symbols for the conjunction: \verb|/\|
and \verb|&&|, and similarly for disjunction. They are logically
equivalent, but may be treated slightly differently by some
transformations. For instance, \texttt{split} transforms the goal
\verb|A /\ B| into subgoals \verb|A| and \verb|B|, whereas it transforms
\verb|A && B| into subgoals \verb|A| and \verb|A -> B|. Similarly, it
transforms \verb!not (A || B)! into subgoals \verb|not A| and
\verb|not ((not A) /\ B)|.

\subsection{Theories}

The syntax for theories is given on Figure~\ref{fig:bnf:theorya} and~\ref{fig:bnf:theoryb}.

\begin{figure}
  \begin{center}\framebox{\input{./theory_bnf.tex}}\end{center}
  \caption{Syntax for theories (part 1).}
\label{fig:bnf:theorya}
\end{figure}

\begin{figure}
  \begin{center}\framebox{\input{./theory2_bnf.tex}}\end{center}
  \caption{Syntax for theories (part 2).}
\label{fig:bnf:theoryb}
\end{figure}

\subsection{Files}

A \why input file is a (possibly empty) list of theories.
\begin{center}\framebox{\input{./why_file_bnf.tex}}\end{center}


%%%%%%%%%%%%%%%%%%%%%%%%%%%%%%%%%%%%%%%%%%%%%%%%%%%%%%%%%%%%%%%%%%%%%%%%%%%%%%
\clearpage
\section{The \whyml Language}\label{sec:syntax:whyml}

\subsection{Specification}

The syntax for specification clauses in programs
is given in Figure~\ref{fig:bnf:spec}.
\begin{figure}
  \begin{center}\framebox{\input{./spec_bnf.tex}}\end{center}
  \caption{Specification clauses in programs.}
\label{fig:bnf:spec}
\end{figure}
Within specifications, terms are extended with new constructs
\verb|old| and \verb|at|:
\begin{center}\framebox{\input{./term_old_at_bnf.tex}}\end{center}
Within a postcondition, $\verb|old|~t$ refers to the value of term $t$
in the prestate. Within the scope of a code mark $L$,
the term $\verb|at|~t~\verb|'|L$ refers to the value of term $t$ at the program
point corresponding to $L$.

\subsection{Expressions}

The syntax for program expressions is given in
Figure~\ref{fig:bnf:expra} and~Figure~\ref{fig:bnf:exprb}.
\begin{figure}
  \begin{center}\framebox{\input{./expr_bnf.tex}}\end{center}
  \caption{Syntax for program expressions (part 1).}
\label{fig:bnf:expra}
\end{figure}

\begin{figure}
  \begin{center}\framebox{\input{./expr2_bnf.tex}}\end{center}
  \caption{Syntax for program expressions (part 2).}
\label{fig:bnf:exprb}
\end{figure}

In applications, arguments are evaluated from right to left.
This includes applications of infix operators, with the only exception of
lazy operators \verb|&&| and \verb+||+ that evaluate from left to
right, lazily.


% In the following we describe the informal semantics of each
% constructs, and provide the corresponding rule for computing the
% weakest precondition.


% \subsubsection{Constant Expressions, Unary and Binary Operators}


% Integer and real constants are as in the logic language, as weel as the unary and binary operators.


% \subsubsection{Array accesses and updates, fields access and updates}

% \todo{}

% \subsubsection{Let binding, sequences}

% \todo{}

% \subsubsection{Function definition}

% \todo{fun, let rec}

% \subsubsection{Function call}

% \todo{}

% \subsubsection{Exception throwing and catching}

% \todo{raise, try with end}

% \subsubsection{Conditional expression, pattern matching}

% \todo{if then else. Discuss standard WP versus fast WP}

% \subsubsection{Iteration Expressions}

% There are three kind of expressions for iterating: bounded, unbounded and infinite.

% \begin{itemize}
% \item A bounded iteration has the form
% \begin{flushleft}\ttfamily
%   for $i$=$a$ to $b$ do invariant \{ $I$ \} $e$ done
% \end{flushleft}
% Expressions $a$ and $b$ are evaluated first and only once, then expression $e$ si evaluated successively for $i$ from $a$ to $b$ included. Nothing is executed if $a > b$. The invariant $I$ must hold at each iteration including before entering the loop and when leaving it. The rule for computing WP is as follows:
% \begin{eqnarray*}
%   WP(\texttt{for} \ldots, Q) &=& I(a) \land \\
% && \forall \vec{w} (\forall i, a \leq i \leq b \land I(i) \rightarrow WP(e,I(i+1))) \land (I(b+1) \rightarrow Q)
% \end{eqnarray*}
% where $\vec{w}$ is the set of references modified by $e$.

% A downward bounded iteration is also available, under the form
% \begin{flushleft}\ttfamily
%   for $i$=$a$ downto $b$ do invariant \{ $I$ \} $e$ done
% \end{flushleft}

% \item An unbounded iteration has the form
% \begin{flushleft}\ttfamily
%   while $c$ do invariant \{ $I$ \} $e$ done
% \end{flushleft}
% it iterates the loop body $e$ until the condition $c$ becomes false. 
% \begin{eqnarray*}
%   WP(\texttt{while} \ldots, Q) &=& I \land \\
% && \forall \vec{w} (c \land I \rightarrow WP(e,I)) \land (\neg c \land I \rightarrow Q)
% \end{eqnarray*}
% where $\vec{w}$ is the set of references modified by $e$.

% Additionally, such a loop can be given a variant $v$, a quantity that must decreases ar each iteration, so as to prove its termination.


% \item An infinite iteration has the form
% \begin{flushleft}\ttfamily
%   loop invariant \{ $I$ \} $e$ end
% \end{flushleft}
% it iterates the loop forever, hence the only way to exit such a loop is to raise an exception.
% \begin{eqnarray*}
%   WP(\texttt{loop} \ldots, Q \mid Exc \Rightarrow R) &=& I \land \\
% && \forall \vec{w} (I \rightarrow WP(e,I)) \land (I \rightarrow WP(e,Exc \Rightarrow R))
% \end{eqnarray*}
% \end{itemize}

% \subsubsection{Assertions, Code Contracts}

% There are several form of expressions for inserting annotations in the code.
% The first form of those are the \emph{assertions} which have the form
% \begin{flushleft}\ttfamily
%   \textsl{keyword} \{ \textsl{proposition} \}
% \end{flushleft}
% where \textsl{keyword} is either \texttt{assert}, \texttt{assume} or
% \texttt{check}. They all state that the given proposition holds at the given program point. The differences are:
% \begin{itemize}
% \item \texttt{assert} requires to prove that the proposition holds, and then make it available in the context of the remaining of the code
% \item \texttt{check} requires to prove that the proposition holds, but
%   does not make it visible in the remaining
% \item \texttt{assume} assumes that the proposition holds and make it
%   visible in the context of the remaining code, without requiring to
%   prove it. It acts like an axiom, but within a program.
% \end{itemize}
% The corresponding rules for computing WP are as follows:
% \begin{eqnarray*}
%   WP(\texttt{assert} \{ P \}, Q) &=& P \mathop{\&\&} Q = P \land (P \rightarrow Q)\\
%   WP(\texttt{check} \{ P \}, Q) &=& P \land Q \\
%   WP(\texttt{assume} \{ P \}, Q) &=& P \rightarrow Q
% \end{eqnarray*}

% The other forms of code contracts allow to abstract a piece of code by specifications.
% \begin{itemize}
% \item $\texttt{any}~\{ P \}~\tau~\epsilon~\{ Q \}$ is a
%   non-deterministic expression that requires the precondition $P$ to
%   hold, then makes some side effects $\epsilon$, and returns any value
%   of type $\tau$ such that $Q$ holds. This construct acts as an axiom
%   in the sense that it does not check whether there exists any program
%   that can effectively establish the post-condition (similarly as the
%   introduction of a \texttt{val} at the global level).
% \item $\texttt{abstract}~e~\{ Q \}$ makes sure that the evaluation of
%   expression $e$ establishes the post-condition $Q$, and then abstract
%   away the program state by the post-condition $Q$ (similarly to the
%   \texttt{any} construct).
% \end{itemize}
% The corresponding rules for computing WP are as follows:
% \[
% \begin{array}{l}
%   WP(\texttt{any}~\{ P \}~\tau~\epsilon~\{ Q \mid exn_i \Rightarrow R_i \} ,
%   Q'  exn_i \Rightarrow R'_i) = \\
%   \qquad\qquad P \mathop{\&\&} \forall result, \epsilon.
%   (Q \rightarrow Q') \land (R_i \rightarrow R'_i) \\
%   WP(\texttt{abstract}~e~\{ Q \mid exn_i \Rightarrow R_i \} ,
%   Q' \mid exn_i \Rightarrow R'_i) = \\
%   \qquad\qquad WP(e,Q \mid exn_i \Rightarrow R_i) \land
%   \forall result, \epsilon, Q \rightarrow Q' \land R_i \rightarrow R'_i
% \end{array}
% \]

% \subsubsection{Labels, Operators \texttt{old} and \texttt{at}}

% \todo{Labels, old, at}

\subsection{Modules}

The syntax for modules is given in Figure~\ref{fig:bnf:module}.
\begin{figure}
  \begin{center}\framebox{\input{./module_bnf.tex}}\end{center}
  \caption{Syntax for modules.}
\label{fig:bnf:module}
\end{figure}
Any declaration which is accepted in a theory is also accepted in a
module. Additionally, modules can introduce record types with mutable
fields and declarations which are specific to programs (global
variables, functions, exceptions).

\subsection{Files}

A \whyml input file is a (possibly empty) list of theories and modules.
\begin{center}\framebox{\input{./whyml_file_bnf.tex}}\end{center}
A theory defined in a \whyml file can only be used within that
file. If a theory is supposed to be reused from other files, be they
\why or \whyml files, it should be defined in a \why file.


\section{The \why Standard Library}
\label{sec:library}\index{standard library}\index{library}

The \why standard library provides general-purpose theories and
modules, to be used in logic and/or programs.
It can be browsed on-line at \url{http://why3.lri.fr/stdlib/}.
Each file contains one or several theories and/or modules.
To \texttt{use} or \texttt{clone} a theory/module \texttt{T} from file
\texttt{file}, use the syntax \texttt{file.T}, since \texttt{file} is
available in \why's default load path. For instance, the theory of
integers and the module of references are imported as follows:

\begin{whycode}
  use import int.Int
  use import ref.Ref
\end{whycode}


%%% Local Variables:
%%% mode: latex
%%% TeX-PDF-mode: t
%%% TeX-master: "manual"
%%% End:



\chapter{Executing \whyml Programs}
\label{chap:exec}\index{whyml@\whyml}

This chapter shows how \whyml code can be executed, either by being
interpreted or compiled to some existing programming language.

\begin{latexonly}
Let us consider the program in Figure~\ref{fig:MaxAndSum}
on page~\pageref{fig:MaxAndSum} that computes the maximum and the sum
of an array of integers.
\end{latexonly}
\begin{htmlonly}
Let us consider the program of Section~\ref{sec:MaxAndSum} that computes
the maximum and the sum of an array of integers.
\end{htmlonly}
Let us assume it is contained in a file \texttt{maxsum.mlw}.

\section{Interpreting \whyml Code}
\label{sec:execute}
\index{execute@\texttt{execute}}\index{interpretation!of \whyml}
\index{testing \whyml code}

To test function \texttt{max\_sum}, we can introduce a \whyml test function
in module \texttt{MaxAndSum}
\begin{whycode}
  let test () =
    let n = 10 in
    let a = make n 0 in
    a[0] <- 9; a[1] <- 5; a[2] <- 0; a[3] <- 2;  a[4] <- 7;
    a[5] <- 3; a[6] <- 2; a[7] <- 1; a[8] <- 10; a[9] <- 6;
    max_sum a n
\end{whycode}
and then we use the \texttt{execute} command to interpret this function,
as follows:
\begin{verbatim}
> why3 execute maxsum.mlw MaxAndSum.test
Execution of MaxAndSum.test ():
     type: (int, int)
   result: (45, 10)
  globals:
\end{verbatim}
We get the expected output, namely the pair \texttt{(45, 10)}.

\section{Compiling \whyml to OCaml}
\label{sec:extract}
\index{OCaml}\index{extraction}
\index{extract@\texttt{extract}}

An alternative to interpretation is to compile \whyml to OCaml.
We do so using the \texttt{extract} command, as follows:
\begin{verbatim}
> mkdir dir
> why3 extract -D ocaml64 maxsum.mlw -o dir
\end{verbatim}
The \texttt{extract} command requires the name of a driver, which indicates
how theories/modules from the \why standard library are translated to
OCaml. Here we assume a 64-bit architecture and thus we pass
\texttt{ocaml64}. On a 32-bit architecture, we would use
\texttt{ocaml32} instead. Extraction also requires a target directory
to be specified using option \verb+-o+. Here we pass a freshly created
directory \texttt{dir}.

Directory \texttt{dir} is now populated with a bunch of OCaml files,
among which we find a file \texttt{maxsum\_\_MaxAndSum.ml} containing
the OCaml code for functions \texttt{max\_sum} and \texttt{test}.

To compile it, we create a file \texttt{main.ml}
containing a call to \texttt{test}, that is, for example,
\begin{whycode}
  let (s,m) = test () in
  Format.printf "sum=%s, max=%s@."
    (Why3__BigInt.to_string s) (Why3__BigInt.to_string m)
\end{whycode}
and we pass both files \texttt{maxsum\_\_MaxAndSum.ml} and
\texttt{main.ml} to the OCaml compiler:
\begin{verbatim}
> cd dir
> ocamlopt zarith.cmxa why3extract.cmxa maxsum__MaxAndSum.ml main.ml
\end{verbatim}
OCaml code extracted from \why must be linked with the library
\texttt{why3extract.cmxa} that is shipped with \why. It is typically
stored in subdirectory \texttt{why3} of the OCaml standard library.
Depending on the way \why was installed, it depends either on library
\texttt{nums.cmxa} or \texttt{zarith.cmxa} for big integers. Above we
assumed the latter. It is likely that additional options \texttt{-I}
must be passed to the OCaml compiler for libraries
\texttt{zarith.cmxa} and \texttt{why3extract.cmxa} to be found.
For instance, it could be
\begin{verbatim}
> ocamlopt -I `ocamlfind query zarith` zarith.cmxa \
           -I `why3 --print-libdir`/why3 why3extract.cmxa \
           ...
\end{verbatim}


%%% Local Variables:
%%% mode: latex
%%% TeX-PDF-mode: t
%%% TeX-master: "manual"
%%% End:


% maintaining library.tex up to date is hopeless
% \input{library.tex}


\chapter{Interactive Proof Assistants}


% ... We then provide specific information about some ITPs.

\section{Using an Interactive Proof Assistant to Discharge Goals}

Instead of calling an automated theorem prover to discharge a goal,
\why offers the possibility to call an interactive theorem prover
instead. In that case, the interaction is decomposed into two distinct
phases:
\begin{itemize}
\item Edition of a proof script for the goal, typically inside a proof editor
  provided by the external interactive theorem prover;
\item Replay of an existing proof script.
\end{itemize}
An example of such an interaction is given in the tutorial
section~\ref{sec:gui}.

Some proof assistants offer more than one possible editor, \eg a
choice between the use of a dedicated editor and the use of the Emacs
editor and the ProofGeneral mode. Selection of the preferred mode can
be made in \texttt{why3ide} preferences, under the ``Editors'' tab.

\section{Theory Realizations}
\label{sec:realizations}

Given a \why theory, one can use a proof assistant to make a
\emph{realization} of this theory, that is to provide definitions for
some of its uninterpreted symbols and proofs for some of its
axioms. This way, one can show the consistency of an axiomatized
theory and/or make a connection to an existing library (of the proof
assistant) to ease some proofs.
%Currently, realizations are supported for the proof assistants Coq and PVS.

\subsection{Generating a realization}

Generating the skeleton for a theory is done by passing to the
\texttt{realize} command a driver suitable for realizations, the names of
the theories to realize, and a target directory.
\index{realize@\texttt{realize}}

\begin{verbatim}
why3 realize -D path/to/drivers/prover-realize.drv
             -T env_path.theory_name -o path/to/target/dir/
\end{verbatim}
\index{driver@\verb+--driver+}
\index{theory@\verb+--theory+}

The theory is looked into the files from the environment, \eg the standard
library. If the theory is stored in a different location, option \texttt{-L}
should be used.

The name of the generated file is inferred from the theory name. If the
target directory already contains a file with the same name, \why
extracts all the parts that it assumes to be user-edited and merges them in
the generated file.

Note that \why does not track dependencies between realizations and
theories, so a realization will become outdated if the corresponding
theory is modified.
It is up to the user to handle such dependencies, for instance using a
\texttt{Makefile}.

\subsection{Using realizations inside proofs}

If a theory has been realized, the \why printer for the corresponding prover
will no longer output declarations for that theory but instead simply put
a directive to load the realization. In order to tell the printer
that a given theory is realized, one has to add a meta declaration in the
corresponding theory section of the driver.
\index{driver file}

\begin{verbatim}
theory env_path.theory_name
  meta "realized_theory" "env_path.theory_name", "optional_naming"
end
\end{verbatim}

The first parameter is the theory name for \why. The second
parameter, if not empty, provides a name to be used inside generated
scripts to point to the realization, in case the default name is not
suitable for the interactive prover.
\index{realized_theory@\verb+realized_theory+}

\subsection{Shipping libraries of realizations}

While modifying an existing driver file might be sufficient for local
use, it does not scale well when the realizations are to be shipped to
other users. Instead, one should create two additional files: a
configuration file that indicates how to modify paths, provers, and
editors, and a driver file that contains only the needed
\verb+meta "realized_theory"+ declarations. The configuration file should be as
follows.
\index{configuration file}

\begin{verbatim}
[main]
loadpath="path/to/theories"

[prover_modifiers]
name="Coq"
option="-R path/to/vo/files Logical_directory"
driver="path/to/file/with/meta.drv"

[editor_modifiers coqide]
option="-R path/to/vo/files Logical_directory"

[editor_modifiers proofgeneral-coq]
option="--eval \"(setq coq-load-path (cons '(\\\"path/to/vo/files\\\" \
  \\\"Logical_directory\\\") coq-load-path))\""
\end{verbatim}

This configuration file can be passed to \why thanks to the
\verb+--extra-config+ option.
\index{extra-config@\verb+--extra-config+}
\index{prover_modifiers@\verb+prover_modifiers+}
\index{editor_modifiers@\verb+editor_modifiers+}
\index{option@\verb+option+}
\index{driver@\verb+driver+}


\input{./coq.tex}

\subsection{Coq Tactic}
\label{sec:coqtactic}

\why provides a Coq tactic to call external theorem provers as oracles.

\subsubsection{Installation}

You need Coq version 8.4 or greater. If this is the case, \why's
configuration detects it, then compiles and installs the Coq tactic.
The Coq tactic is installed in
\begin{center}
  \textit{why3-lib-dir}\texttt{/coq-tactic/}
\end{center}
where \textit{why3-lib-dir} is \why's library directory, as reported
by \verb+why3 --print-libdir+. This directory
is automatically added to Coq's load path if you are
calling Coq via \why (from \texttt{why3 ide}, \texttt{why3 replay},
etc.). If you are calling Coq by yourself, you need to add
this directory to Coq's load path, either using Coq's command line
option \texttt{-I} or by adding
\begin{center}
  \verb+Add LoadPath "+\textit{why3-lib-dir}\verb+/coq-tactic/".+
\end{center}
to your \texttt{\~{}/.coqrc} resource file.

\subsubsection{Usage}

The Coq tactic is called \texttt{why3} and is used as follows:
\begin{center}
  \texttt{why3} \verb+"+\textit{prover-name}\verb+"+
  $[$\texttt{timelimit} \textit{n}$]$.
\end{center}
The string \textit{prover-name} identifies one of the automated theorem provers
supported by \why, as reported by \verb+why3 --list-provers+
(interactive provers excluded).
\index{list-provers@\verb+--list-provers+}
The current goal is then translated to \why's logic and the prover is
called. If it reports the goal to be valid, then Coq's \texttt{admit}
tactic is used to assume the goal. The prover is called with a time
limit in seconds as given by \why's configuration file
(see Section~\ref{sec:whyconffile}). A different value may be given
using the \texttt{timelimit} keyword.

\subsubsection{Error messages.} The following errors may be reported by
the Coq tactic.
\begin{description}
\item[\texttt{Not a first order goal}]\emptyitem
  The Coq goal could not be translated to \why's logic.
\item[\texttt{Timeout}]\emptyitem
  There was no answer from the prover within the given time limit.
\item[\texttt{Don't know}]\emptyitem
  The prover stopped without validating the goal.
\item[\texttt{Invalid}]\emptyitem
  The prover stopped, reporting the goal to be invalid.
\item[\texttt{Failure}]\emptyitem
  The prover failed. Depending on the message that follows, you may
  want to file a bug report, either to the \why\ developers or to the
  prover developers.
\end{description}

%%% Local Variables:
%%% mode: latex
%%% compile-command: "make -C .. doc"
%%% TeX-PDF-mode: t
%%% TeX-master: "manual"
%%% End:


\section{Isabelle/HOL}
\label{sec:isabelle}

\index{Isabelle proof assistant}

When using Isabelle from \why, files generated from \why theories and
goals are stored in a dedicated XML format. Those files should not be
edited. Instead, the proofs must be completed in a file with the same
name and extension \texttt{.thy}. This is the file that is opened when
using ``Edit'' action in \texttt{why3ide}.

\subsection{Installation}

You need version Isabelle2014. Former versions are not supported.

Isabelle must be installed before compiling \why. After compilation
and installation of \why, you must manually add the path
\begin{verbatim}
<Why3 lib dir>/isabelle
\end{verbatim}
into either the user file
\begin{verbatim}
.isabelle/Isabelle2014/etc/components
\end{verbatim}
or the system-wide file
\begin{verbatim}
<Isabelle install dir>/etc/components
\end{verbatim}

\subsection{Usage}

The most convenient way to call Isabelle for discharging a \why goal
is to start the Isabelle/jedit interface in server mode. In this mode,
one must start the server once, before launching \texttt{why3ide},
using
\begin{verbatim}
isabelle why3_jedit
\end{verbatim}
Then, inside a \texttt{why3ide} session, any use of ``Edit'' will
transfer the file to the already opened instance of jEdit. When the
proof is completed, the user must send back the edited proof to
\texttt{why3ide} by closing the opened buffer, typically by hitting
\texttt{Ctrl-w}.

\subsection{Realizations}

Realizations must be designed in some \texttt{.thy} as follows.
The realization file corresponding to some \why file \texttt{f.why}
should have the following form.
\begin{verbatim}
theory Why3_f
imports Why3_Setup
begin

section {* realization of theory T *}

why3_open "f/T.xml"

why3_vc <some lemma>
<proof>

why3_vc <some other lemma> by proof

[...]

why3_end
\end{verbatim}

See directory \texttt{lib/isabelle} for examples.


%%% Local Variables:
%%% mode: latex
%%% TeX-PDF-mode: t
%%% TeX-master: "manual"
%%% End:


\input{./pvs.tex}


%%% Local Variables:
%%% mode: latex
%%% TeX-PDF-mode: t
%%% TeX-master: "manual"
%%% End:



% \chapter{Complete API documentation} *)
% \label{chap:apidoc} *)

% \input{./apidoc.tex} *)

\chapter{Technical Informations}


\section{Structure of Session Files}

The proof session state is stored in an XML file named
\texttt{\textsl{<dir>}/why3session.xml}, where \texttt{\textsl{<dir>}}
is the directory of the project.
The XML file follows the DTD given in \texttt{share/why3session.dtd} and reproduced below.
\lstinputlisting{../share/why3session.dtd}


\section{Prover Detection}
\label{sec:proverdetecttiondata}

All the necessary data configuration for the automatic detection of
installed provers is stored in the file
\texttt{provers-detection-data.conf} typically located in directory
\verb|/usr/local/share/why3| after installation. The contents of this
file is reproduced below.
%BEGIN LATEX
{\footnotesize
%END LATEX
\lstinputlisting{../share/provers-detection-data.conf}
%BEGIN LATEX
}
%END LATEX

\section{The \texttt{why3.conf} Configuration File}
\label{sec:whyconffile}
\index{why3.conf@\texttt{why3.conf}}\index{configuration file}

One can use a custom configuration file. The \why
tools look for it in the following order:
\begin{enumerate}
\item the file specified by the \texttt{-C} or \texttt{-{}-config} options,
\item the file specified by the environment variable
  \texttt{WHY3CONFIG} if set,
\item the file \texttt{\$HOME/.why3.conf}
  (\texttt{\$USERPROFILE/.why3.conf} under Windows) or, in the case of
  local installation, \texttt{why3.conf} in the top directory of \why sources.
\end{enumerate}
If none of these files exist, a built-in default configuration is used.

The configuration file is a human-readable text file, which consists
of association pairs arranged in sections.
%BEGIN LATEX
Figure~\ref{fig:why3conf} shows an example of configuration file.
%END LATEX
%HEVEA Below is an example of configuration file.

%BEGIN LATEX
\begin{figure}[p]
{\footnotesize
%END LATEX
\begin{verbatim}
[main]
loadpath = "/usr/local/share/why3/theories"
loadpath = "/usr/local/share/why3/modules"
magic = 14
memlimit = 0
plugin = "/usr/local/lib/why3/plugins/tptp"
plugin = "/usr/local/lib/why3/plugins/genequlin"
plugin = "/usr/local/lib/why3/plugins/hypothesis_selection"
running_provers_max = 4
timelimit = 2

[ide]
default_editor = "editor %f"
error_color = "orange"
goal_color = "gold"
iconset = "boomy"
intro_premises = true
premise_color = "chartreuse"
print_labels = false
print_locs = false
print_time_limit = false
saving_policy = 2
task_height = 404
tree_width = 512
verbose = 0
window_height = 1173
window_width = 1024

[prover]
command = "'why3-cpulimit' 0 %m -s coqtop -batch -I %l/coq-tactic -R %l/coq Why3 -l %f"
driver = "/usr/local/share/why3/drivers/coq.drv"
editor = "coqide"
in_place = false
interactive = true
name = "Coq"
shortcut = "coq"
version = "8.3pl4"

[prover]
command = "'why3-cpulimit' %t %m -s alt-ergo %f"
driver = "/usr/local/share/why3/drivers/alt_ergo_0.93.drv"
editor = ""
in_place = false
interactive = false
name = "Alt-Ergo"
shortcut = "altergo"
shortcut = "alt-ergo"
version = "0.93.1"

[editor coqide]
command = "coqide -I %l/coq-tactic -R %l/coq Why3 %f"
name = "CoqIDE"
\end{verbatim}
%BEGIN LATEX
}
\caption{Sample \texttt{why3.conf} file}
\label{fig:why3conf}
\end{figure}
%END LATEX

A section begins with a header inside square brackets and ends at the
beginning of the next section. The header of a
section can be only one identifier, \texttt{main} and \texttt{ide} in
the example, or it can be composed by a family name and one family
argument, \texttt{prover} is one family name, \texttt{coq} and
\texttt{alt-ergo} are the family argument.

Sections contain associations \texttt{key=value}. A value is either
an integer (\eg \texttt{-555}), a boolean (\texttt{true}, \texttt{false}),
or a string (\eg \texttt{"emacs"}). Some specific keys can be attributed
multiple values and are
thus allowed to occur several times inside a given section. In that
case, the relative order of these associations matter.

\section{Drivers for External Provers}
\label{sec:drivers}

Drivers for external provers are readable files from directory
\texttt{drivers}. Experimented users can modify them to change the way
the external provers are called, in particular which transformations
are applied to goals.

[TO BE COMPLETED LATER]

\section{Transformations}
\label{sec:transformations}

Here is a quick documentation of provided transformations. We give
first the non-splitting ones, \eg those which produce one goal as
result, and others which produces any number of goals.

Notice that the set of available transformations in your own
installation is given by
\begin{verbatim}
why3 --list-transforms
\end{verbatim}
\index{list-transforms@\verb+--list-transforms+}

\subsection{Non-splitting transformations}

\begin{description}

\item[eliminate\_algebraic] replaces algebraic data types by first-order
definitions~\cite{paskevich09rr}.

\item[eliminate\_builtin] removes definitions of symbols that are
  declared as builtin in the driver, \ie with a ``syntax'' rule.
\item[eliminate\_definition\_func]
  replaces all function definitions with axioms.
\item[eliminate\_definition\_pred]
  replaces all predicate definitions with axioms.
\item[eliminate\_definition]
  applies both transformations above.
\item[eliminate\_mutual\_recursion]
  replaces mutually recursive definitions with axioms.
\item[eliminate\_recursion]
  replaces all recursive definitions with axioms.

\item[eliminate\_if\_term] replaces terms of the form \texttt{if
    formula then t2 else t3} by lifting them at the level of formulas.
  This may introduce \texttt{if then else } in formulas.

\item[eliminate\_if\_fmla] replaces formulas of the form \texttt{if f1 then f2
  else f3} by an equivalent formula using implications and other
  connectives.

\item[eliminate\_if]
  applies both transformations above.

\item[eliminate\_inductive] replaces inductive predicates by
  (incomplete) axiomatic definitions, \ie construction axioms and
  an inversion axiom.

\item[eliminate\_let\_fmla]
  eliminates \texttt{let} by substitution, at the predicate level.

\item[eliminate\_let\_term]
  eliminates \texttt{let} by substitution, at the term level.

\item[eliminate\_let]
  applies both transformations above.

% \item[encoding\_decorate\_mono]

% \item[encoding\_enumeration]

\item[encoding\_smt]
  encodes polymorphic types into monomorphic type~\cite{conchon08smt}.

\item[encoding\_tptp]
  encodes theories into unsorted logic. %~\cite{cruanes10}.

% \item[filter\_trigger] *)

% \item[filter\_trigger\_builtin] *)

% \item[filter\_trigger\_no\_predicate] *)

% \item[hypothesis\_selection] *)
%   Filter hypothesis of goals~\cite{couchot07ftp,cruanes10}. *)

\item[inline\_all]
  expands all non-recursive definitions.

\item[inline\_goal] expands all outermost symbols of the goal that
  have a non-recursive definition.

\item[inline\_trivial]
  removes definitions of the form
\begin{whycode}
function  f x_1 ... x_n = (g e_1 ... e_k)
predicate p x_1 ... x_n = (q e_1 ... e_k)
\end{whycode}
when each $e_i$ is either a ground term or one of the $x_j$, and
each $x_1 \dots x_n$ occurs at most once in all the $e_i$.

\item[introduce\_premises] moves antecedents of implications and
  universal quantifications of the goal into the premises of the task.

% \item[remove\_triggers] *)
%   removes the triggers in all quantifications. *)

\item[simplify\_array] automatically rewrites the task using the lemma
  \verb|Select_eq| of theory \verb|array.Array|.

\item[simplify\_formula] reduces trivial equalities $t=t$ to true and
  then simplifies propositional structure: removes true, false, simplifies
  $f \land f$ to $f$, etc.

\item[simplify\_recursive\_definition] reduces mutually recursive
  definitions if they are not really mutually recursive, \eg
\begin{whycode}
function f : ... = .... g ...
with g : ... = e
\end{whycode}
becomes
\begin{whycode}
function g : ... = e
function f : ... = ... g ...
\end{whycode}
if $f$ does not occur in $e$.

\item[simplify\_trivial\_quantification]
  simplifies quantifications of the form
\begin{verbatim}
forall x, x=t -> P(x)
\end{verbatim}
or
\begin{verbatim}
forall x, t=x -> P(x)
\end{verbatim}
when $x$ does not occur in $t$ into
\begin{verbatim}
P(t)
\end{verbatim}
  More generally, it applies this simplification whenever $x=t$ appears
  in a negative position.

\item[simplify\_trivial\_quantification\_in\_goal]
  is the same as above but it applies only in the goal.

\item[split\_premise]
  splits conjunctive premises.

\end{description}

\subsection{Splitting transformations}

\begin{description}

\item[full\_split\_all]
  performs both \texttt{split\_premise} and \texttt{full\_split\_goal}.

\item[full\_split\_goal] puts the goal in a conjunctive form,
  returns the corresponding set of subgoals. The number of subgoals
  generated may be exponential in the size of the initial goal.

\item[simplify\_formula\_and\_task] is the same as \texttt{simplify\_formula}
  but it also removes the goal if it is equivalent to true.

\item[split\_all]
  performs both \texttt{split\_premise} and \texttt{split\_goal}.

\item[split\_goal] if the goal is a conjunction of goals, returns the
  corresponding set of subgoals. The number of subgoals generated is linear in
  the size of the initial goal.

\item[split\_intro]
  moves the antecedents into the premises when a goal is an implication.

\end{description}


%%% Local Variables:
%%% mode: latex
%%% TeX-PDF-mode: t
%%% TeX-master: "manual"
%%% End:



\part{Appendix}

\appendix

\chapter{Release Notes}

\section{Release Notes for version 0.80: syntax changes w.r.t. 0.73}

The syntax of \whyml programs changed in release 0.80. 
The table in Figure~\ref{fig:syntax080} summarizes the changes.

\begin{figure}[thbp]
  \centering
\begin{tabular}{|p{0.45\textwidth}|p{0.45\textwidth}|}
\hline
\textbf{version 0.73} & \textbf{version 0.80} \\
\hline
\ttfamily
type t = \{| field~:~int |\}
&
\ttfamily
type t = \{ field~:~int \}
\\
\hline
\ttfamily
\{| field = 5 |\}
&
\ttfamily
\{ field = 5 \}
\\
\hline
\ttfamily
use import module M
&
\ttfamily
use import M
\\
\hline
\ttfamily
let rec f (x:int) (y:int)~:~t \newline
\null~~~~variant \{ t \} with rel = \newline
\null~~~~\{ P \} \newline
\null~~~~e \newline
\null~~~~\{ Q \} \newline
\null~~~~| Exc1 -> \{ R1 \} \newline
\null~~~~| Exc2 n -> \{ R2 \}
&
\ttfamily
let rec f (x:int) (y:int)~:~t \newline
\null~~~~variant \{ t with rel \} \newline
\null~~~~requires \{ P \} \newline
\null~~~~ensures \{ Q \} \newline
\null~~~~raises \{ Exc1 -> R1 \newline
\null~~~~~~~~~~~| Exc2 n -> R2 \} \newline
\null~~~~= e
\\
\hline
\ttfamily
val f (x:int) (y:int)~:\newline
\null~~~~\{ P \} \newline
\null~~~~t \newline
\null~~~~writes a b \newline
\null~~~~\{ Q \} \newline
\null~~~~| Exc1 -> \{ R1 \} \newline
\null~~~~| Exc2 n -> \{ R2 \}
&
\ttfamily
val f (x:int) (y:int)~:~t \newline
\null~~~~requires \{ P \} \newline
\null~~~~writes \{ a, b \} \newline
\null~~~~ensures \{ Q \} \newline
\null~~~~raises \{ Exc1 -> R1 \newline
\null~~~~~~~~~~~| Exc2 n -> R2 \}
\\
\hline
\ttfamily
val f~:~x:int -> y:int ->\newline
\null~~~~\{ P \} \newline
\null~~~~t \newline
\null~~~~writes a b \newline
\null~~~~\{ Q \} \newline
\null~~~~| Exc1 -> \{ R1 \} \newline
\null~~~~| Exc2 n -> \{ R2 \}
&
\ttfamily
val f (x y:int)~:~t \newline
\null~~~~requires \{ P \} \newline
\null~~~~writes \{ a, b \} \newline
\null~~~~ensures \{ Q \} \newline
\null~~~~raises \{ Exc1 -> R1 \newline
\null~~~~~~~~~~~| Exc2 n -> R2 \}
\\
\hline
\ttfamily
abstract e \{ Q \}
&
\ttfamily
abstract e ensures \{ Q \}
\\
\hline
\end{tabular}
\caption{Syntax changes from version 0.73 to version 0.80}
\label{fig:syntax080}
\end{figure}

\section{Summary of Changes w.r.t. Why 2}

The main new features with respect to Why 2.xx
are the following.
\begin{enumerate}
\item Completely redesigned input syntax for logic declarations
  \begin{itemize}
  \item new syntax for terms and formulas
  \item enumerated and algebraic data types, pattern matching
  \item recursive definitions of logic functions and predicates, with
    termination checking
  \item inductive definitions of predicates
  \item declarations are structured in components called ``theories'',
    which can be reused and instantiated
  \end{itemize}

\item More generic handling of goals and lemmas to prove
  \begin{itemize}
  \item concept of proof task
  \item generic concept of task transformation
  \item generic approach for communicating with external provers
  \end{itemize}

\item Source code organized as a library with a documented API, to
  allow access to \why features programmatically.

\item GUI with new features with respect to the former GWhy
  \begin{itemize}
  \item session save and restore
  \item prover calls in parallel
  \item splitting, and more generally applying task transformations,
    on demand
  \item ability to edit proofs for interactive provers (Coq only for
    the moment) on any subtask
  \end{itemize}

\item Extensible architecture via plugins
  \begin{itemize}
  \item users can define new transformations
  \item users can add connections to additional provers
  \end{itemize}
\end{enumerate}

% \begin{itemize}
% \item New syntax for terms and formulas
% \item Algebraic data types, pattern matching
% \item Recursive definitions
% \item Inductive predicates
% \item Declaration encapsulated in theories. Using and cloning theories.
% \item Concept of proof task transformation
% \item Generic communication with provers
% \item OCaml library with documented API
% \item New GUI with session save and restore
% % \item New syntax for programs, new VC generator, intentionaly left *)
% %   undocumented, since the syntax is likely to evolve significantly in *)
% %   the future. Examples are available in \texttt{examples/programs}. *)
% \end{itemize}

\bibliographystyle{abbrv}
\bibliography{manual}
%\input{biblio-demons}


% \cleardoublepage
% \input{glossary.tex}

\cleardoublepage
\listoffigures
\cleardoublepage
\printindex

\end{document}

%%% Local Variables:
%%% mode: latex
%%% TeX-PDF-mode: t
%%% TeX-master: t
%%% End:
