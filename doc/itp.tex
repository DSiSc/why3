
\chapter{Interactive Proof Assistants}


% ... We then provide specific information about some ITPs.

\section{Using an Interactive Proof Assistant to Discharge Goals}

Instead of calling an automated theorem prover to discharge a goal,
\why offers the possibility to call an interactive theorem prover
instead. In that case, the interaction is decomposed into two distinct
phases:
\begin{itemize}
\item Edition of a proof script for the goal, typically inside a proof editor
  provided by the external interactive theorem prover;
\item Replay of an existing proof script.
\end{itemize}
An example of such an interaction is given in the tutorial
section~\ref{sec:gui}.

Some proof assistants offer more than one possible editor, e.g. a
choice between the use of a dedicated editor and the use of the Emacs
editor and the ProofGeneral mode. Selection of the preferred mode can
be made in \texttt{why3ide} preferences, under the ``Editors'' tab.

\section{Theory Realizations}
\label{sec:realizations}

Given a \why theory, one can use a proof assistant to make a
\emph{realization} of this theory, that is to provide definitions for
some of its uninterpreted symbols and proofs for some of its
axioms. This way, one can show the consistency of an axiomatized
theory and/or make a connection to an existing library (of the proof
assistant) to ease some proofs.
%Currently, realizations are supported for the proof assistants Coq and PVS.

\subsection{Generating a realization}

Generating the skeleton for a theory is done by passing to the
\texttt{realize} command a driver suitable for realizations, the names of
the theories to realize, and a target directory.
\index{realize@\texttt{realize}}

\begin{verbatim}
why3 realize -D path/to/drivers/prover-realize.drv
             -T env_path.theory_name -o path/to/target/dir/
\end{verbatim}
\index{driver@\verb+--driver+}
\index{theory@\verb+--theory+}

The theory is looked into the files from the environment, \eg the standard
library. If the theory is stored in a different location, option \texttt{-L}
should be used.

The name of the generated file is inferred from the theory name. If the
target directory already contains a file with the same name, \why
extracts all the parts that it assumes to be user-edited and merges them in
the generated file.

Note that \why does not track dependencies between realizations and
theories, so a realization will become outdated if the corresponding
theory is modified.
It is up to the user to handle such dependencies, for instance using a
\texttt{Makefile}.

\subsection{Using realizations inside proofs}

If a theory has been realized, the \why printer for the corresponding prover
will no longer output declarations for that theory but instead simply put
a directive to load the realization. In order to tell the printer
that a given theory is realized, one has to add a meta declaration in the
corresponding theory section of the driver.
\index{driver file}

\begin{verbatim}
theory env_path.theory_name
  meta "realized_theory" "env_path.theory_name", "optional_naming"
end
\end{verbatim}

The first parameter is the theory name for \why. The second
parameter, if not empty, provides a name to be used inside generated
scripts to point to the realization, in case the default name is not
suitable for the interactive prover.
\index{realized_theory@\verb+realized_theory+}

\subsection{Shipping libraries of realizations}

While modifying an existing driver file might be sufficient for local
use, it does not scale well when the realizations are to be shipped to
other users. Instead, one should create two additional files: a
configuration file that indicates how to modify paths, provers, and
editors, and a driver file that contains only the needed
\verb+meta "realized_theory"+ declarations. The configuration file should be as
follows.
\index{configuration file}

\begin{verbatim}
[main]
loadpath="path/to/theories"

[prover_modifiers]
name="Coq"
option="-R path/to/vo/files Logical_directory"
driver="path/to/file/with/meta.drv"

[editor_modifiers coqide]
option="-R path/to/vo/files Logical_directory"

[editor_modifiers proofgeneral-coq]
option="--eval \"(setq coq-load-path (cons '(\\\"path/to/vo/files\\\" \
  \\\"Logical_directory\\\") coq-load-path))\""
\end{verbatim}

This configuration file can be passed to \why thanks to the
\verb+--extra-config+ option.
\index{extra-config@\verb+--extra-config+}
\index{prover_modifiers@\verb+prover_modifiers+}
\index{editor_modifiers@\verb+editor_modifiers+}
\index{option@\verb+option+}
\index{driver@\verb+driver+}


\input{./coq.tex}

\subsection{Coq Tactic}
\label{sec:coqtactic}

\why provides a Coq tactic to call external theorem provers as oracles.

\subsubsection{Installation}

You need Coq version 8.4 or greater. If this is the case, \why's
configuration detects it, then compiles and installs the Coq tactic.
The Coq tactic is installed in
\begin{center}
  \textit{why3-lib-dir}\texttt{/coq-tactic/}
\end{center}
where \textit{why3-lib-dir} is \why's library directory, as reported
by \verb+why3 --print-libdir+. This directory
is automatically added to Coq's load path if you are
calling Coq via \why (from \texttt{why3 ide}, \texttt{why3 replay},
etc.). If you are calling Coq by yourself, you need to add
this directory to Coq's load path, either using Coq's command line
option \texttt{-I} or by adding
\begin{center}
  \verb+Add LoadPath "+\textit{why3-lib-dir}\verb+/coq-tactic/".+
\end{center}
to your \texttt{\~{}/.coqrc} resource file.

\subsubsection{Usage}

The Coq tactic is called \texttt{why3} and is used as follows:
\begin{center}
  \texttt{why3} \verb+"+\textit{prover-name}\verb+"+
  $[$\texttt{timelimit} \textit{n}$]$.
\end{center}
The string \textit{prover-name} identifies one of the automated theorem provers
supported by \why, as reported by \verb+why3 --list-provers+
(interactive provers excluded).
\index{list-provers@\verb+--list-provers+}
The current goal is then translated to \why's logic and the prover is
called. If it reports the goal to be valid, then Coq's \texttt{admit}
tactic is used to assume the goal. The prover is called with a time
limit in seconds as given by \why's configuration file
(see Section~\ref{sec:whyconffile}). A different value may be given
using the \texttt{timelimit} keyword.

\subsubsection{Error messages.} The following errors may be reported by
the Coq tactic.
\begin{description}
\item[\texttt{Not a first order goal}]\emptyitem
  The Coq goal could not be translated to \why's logic.
\item[\texttt{Timeout}]\emptyitem
  There was no answer from the prover within the given time limit.
\item[\texttt{Don't know}]\emptyitem
  The prover stopped without validating the goal.
\item[\texttt{Invalid}]\emptyitem
  The prover stopped, reporting the goal to be invalid.
\item[\texttt{Failure}]\emptyitem
  The prover failed. Depending on the message that follows, you may
  want to file a bug report, either to the \why\ developers or to the
  prover developers.
\end{description}

%%% Local Variables:
%%% mode: latex
%%% compile-command: "make -C .. doc"
%%% TeX-PDF-mode: t
%%% TeX-master: "manual"
%%% End:


\section{Isabelle/HOL}
\label{sec:isabelle}

\index{Isabelle proof assistant}

When using Isabelle from \why, files generated from \why theories and
goals are stored in a dedicated XML format. Those files should not be
edited. Instead, the proofs must be completed in a file with the same
name and extension \texttt{.thy}. This is the file that is opened when
using ``Edit'' action in \texttt{why3ide}.

\subsection{Installation}

You need version Isabelle2014. Former versions are not supported.

Isabelle must be installed before compiling \why. After compilation
and installation of \why, you must manually add the path
\begin{verbatim}
<Why3 lib dir>/isabelle
\end{verbatim}
into either the user file
\begin{verbatim}
.isabelle/Isabelle2014/etc/components
\end{verbatim}
or the system-wide file
\begin{verbatim}
<Isabelle install dir>/etc/components
\end{verbatim}

\subsection{Usage}

The most convenient way to call Isabelle for discharging a \why goal
is to start the Isabelle/jedit interface in server mode. In this mode,
one must start the server once, before launching \texttt{why3ide},
using
\begin{verbatim}
isabelle why3_jedit
\end{verbatim}
Then, inside a \texttt{why3ide} session, any use of ``Edit'' will
transfer the file to the already opened instance of jEdit. When the
proof is completed, the user must send back the edited proof to
\texttt{why3ide} by closing the opened buffer, typically by hitting
\texttt{Ctrl-w}.

\subsection{Realizations}

Realizations must be designed in some \texttt{.thy} as follows.
The realization file corresponding to some \why file \texttt{f.why}
should have the following form.
\begin{verbatim}
theory Why3_f
imports Why3_Setup
begin

section {* realization of theory T *}

why3_open "f/T.xml"

why3_vc <some lemma>
<proof>

why3_vc <some other lemma> by proof

[...]

why3_end
\end{verbatim}

See directory \texttt{lib/isabelle} for examples.


%%% Local Variables:
%%% mode: latex
%%% TeX-PDF-mode: t
%%% TeX-master: "manual"
%%% End:


\input{./pvs.tex}


%%% Local Variables:
%%% mode: latex
%%% TeX-PDF-mode: t
%%% TeX-master: "manual"
%%% End:
